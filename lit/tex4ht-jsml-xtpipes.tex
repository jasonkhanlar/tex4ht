% $Id: tex4ht-jsml-xtpipes.tex 65 2010-11-17 19:16:45Z karl $
% htlatex tex4ht-jsml-xtpipes "xhtml,next,3" "" "-d./"
%
% Copyright (C) 2009-2010 TeX Users Group
% Copyright (C) 2002-2009 Eitan M. Gurari
% Released under LPPL 1.3c+.
% See tex4ht-cpright.tex for license text.

\documentclass{article}
    \Configure{ProTex}{log,<<<>>>,title,list,`,[[]]}
\begin{document}


% $Id: common.tex 65 2010-11-17 19:16:45Z karl $
% A few common TeX definitions for literate sources.  Not installed in runtime.
% 
% Copyright 2009-2010 TeX Users Group
% Copyright 1996-2009 Eitan M. Gurari
%
% This work may be distributed and/or modified under the
% conditions of the LaTeX Project Public License, either
% version 1.3c of this license or (at your option) any
% later version. The latest version of this license is in
%   http://www.latex-project.org/lppl.txt
% and version 1.3c or later is part of all distributions
% of LaTeX version 2005/12/01 or later.
%
% This work has the LPPL maintenance status "maintained".
%
% The Current Maintainer of this work
% is the TeX4ht Project <http://tug.org/tex4ht>.
% 
% If you modify this program, changing the 
% version identification would be appreciated.

\newcount\tmpcnt  \tmpcnt\time  \divide\tmpcnt  60
\edef\temp{\the\tmpcnt}
\multiply\tmpcnt  -60 \advance\tmpcnt  \time

\edef\version{\the\year-\ifnum \month<10 0\fi
  \the\month-\ifnum \day<10 0\fi\the\day
   -\ifnum \temp<10 0\fi \temp
   :\ifnum \tmpcnt<10 0\fi\the\tmpcnt}

% #1 is the first year for Eitan.  The last year is always 2009.  RIP.
\def\CopyYear.#1.{#1-2009}

\input{tex4ht-cpright}

%%%%%%%%%%%%%%%%%%
\part{Script for xtpipes}
%%%%%%%%%%%%%%%%%%


%%%%%%%%%%%%%%%%%%
\section{Outline}
%%%%%%%%%%%%%%%%%%

\AtEndDocument{\OutputCodE\<jsml.4xt\>}

\Needs{"xmllint --valid --noout jsml.4xt"} 


\<jsml.4xt\><<<
<?xml version="1.0" encoding="UTF-8" ?>
<!DOCTYPE xtpipes SYSTEM "xtpipes.dtd" >
<!-- jsml.4xt (`version), generated from `jobname.tex
     Copyright (C) 2009-2010 TeX Users Group
     Copyright (C) `CopyYear.2002. Eitan M. Gurari
`<TeX4ht copyright`> -->
<xtpipes signature="jsml.4xt (`version)">
   <sax content-handler="xtpipes.util.ScriptsManager, tex4ht.GroupMn, tex4ht.JsmlFilter" 
        lexical-handler="xtpipes.util.ScriptsManagerLH" >
     `<numbers in math`>
     `<special sub and super scripts`> 
     `<span frac elements`>
     `<short cut modifiers`>
     `<over and under scripts`>
     `<bold math`>
     `<capital math letters`>
     `<remove multline eqnum cell`> 
     `<inline math`>
     `<display math`>
     `<replace characters`>
     `<remove empty split entries`>
     `<boundaries on theorems`>
     `<br into BREAK`>
     `<short notation for empty elements`>
   </sax>
</xtpipes>
>>>


     `<non short tag br elements`>     

     `<inline math`>
     `<display math`>
     `<measure tables`>




\<inline math\><<<
<script element="span::inline-math" >
  `<compress numeric subscripts`>
  `<set levels for hyper complex fracs`>
  `<set levels for sub and sup scripts`>
  `<set levels for roots`>
  <set name="inline-math" >
     `<open xslt script`>
     `<shared display and inline math 1`>
     `<eliminate inline math narrative`>
     `<close xslt script`>
  </set>
  <xslt name="." xml="." xsl="inline-math" />
  `<eliminate extra math pauses`>
  <set name="inline-math-2" >
     `<open xslt script`>
     `<shared display and inline math 2`>
     `<eliminate inline math narrative 2`>
     `<close xslt script`>
  </set>
  <xslt name="." xml="." xsl="inline-math-2" />
</script> 
>>>


\<display math\><<<
<script element="div::display-math" >
  `<compress numeric subscripts`>
  `<set levels for hyper complex fracs`>
  `<set levels for sub and sup scripts`>
  `<set levels for roots`>
  <set name="display-math" >
     `<open xslt script`>
     `<shared display and inline math 1`>
     `<close xslt script`>
  </set>
  <xslt name="." xml="." xsl="display-math" />
  `<eliminate extra math pauses`>
  <set name="display-math-2" >
     `<open xslt script`>
     `<shared display and inline math 2`>
     `<close xslt script`>
  </set>
  <xslt name="." xml="." xsl="display-math-2" />
  `<set empty elements for the w3 browser`>
</script> 
>>>




\<shared display and inline math 1\><<<
`<remove super mn-group`>
`<sayas punctuation`>
`<sayas digits`>
`<sayas math letters`>
`<get content template`>
`<eliminate baseline script marks`> 
`<clean begin and mid script marks`> 
`<mixed fractions`> 
`<'minus' into 'negative'`> 
`<compress limit script`>
`<remove scrip indicators from primes`> 
`<remove scrip indicators from degree`> 
>>>


\<shared display and inline math 2\><<<
`<get content template`>
`<replace nested baseline script marks`>
`<remove marks on empty scripts`>
`<eliminate begin script marks`> 
>>>




\AtEndDocument{\OutputCodE\<HtJsml.java\>}

\Needs{"
    javac HtJsml.java -d /home/4/gurari/xtpipes.dir/.  
"}

\<HtJsml.java\><<<
package tex4ht;
/* HtJsml.java (`version), generated from `jobname.tex
   Copyright (C) 2009-2010 TeX Users Group
   Copyright (C) `CopyYear.2002. Eitan M. Gurari
`<TeX4ht copyright`> */
import org.w3c.dom.*;
public class HtJsml {
  `<HtJsml utility members`>
  `<static void mnGroup(dom)`>
  `<static void fracLevel(dom)`>
  `<static void scriptLevel(dom)`>
  `<static void rootLevel(dom)`>
}
>>>




%%%%%%%%%%%%%%%%%%
\section{Numbers in Math}
%%%%%%%%%%%%%%%%%%


%%%%%%%%%%%%%
\subsection{Group MN / MO-Punctuation}
%%%%%%%%%%%%%


\AtEndDocument{ 
   \OutputCodE\<GroupMn.java\> 
} 
\Needs{"
    javac GroupMn.java -d /home/4/gurari/xtpipes.dir/. 
"} 
 
\<GroupMn.java\><<< 
package tex4ht;
/* GroupMn.java (`version), generated from `jobname.tex
   Copyright (C) 2009-2010 TeX Users Group
   Copyright (C) `CopyYear.2002. Eitan M. Gurari
`<TeX4ht copyright`> */
import org.xml.sax.helpers.*; 
import org.xml.sax.*; 
import java.io.PrintWriter; 
 
public class GroupMn extends XMLFilterImpl { 
     PrintWriter out = null; 
     boolean inMn = false;
     String ns;
     int level = -1;
   public GroupMn( PrintWriter out, PrintWriter log, boolean trace){ 
     this.out = out; 
   } 
   public void startElement(String ns, String sName, 
                           String qName, Attributes attr) { 
      level++;
      try{ 
        if( inMn ){ 
           if( level == 0 ){ `<consider end of num`> }
        } else { `<consider start of num`> } 
        super.startElement(ns, sName, qName, attr); 
      } catch( Exception e ){
        System.out.println( "--- GroupMn Error 1 --- " + e);
      } 
   }   
   public void endElement(String ns, String sName, String qName){ 
      try{ 
        if( level < 0) { 
          `<endElement: consider end of num`>
        }
        super.endElement(ns, sName, qName); 
      } catch( Exception e ){
        System.out.println( "--- GroupMn Error 2 --- " + e);
      } 
      level--;
   }   
   public void characters(char[] ch, int start, int length){ 
      try{ 
        if ( inMn  && (level < 0) ) {
           String s = new String(ch, start, length);
           if (!s.trim().equals("")) {
             inMn = false;
             super.endElement(ns, "mn-group", "mn-group");
        }  }
        super.characters(ch, start, length); 
      } catch( Exception e ){
        System.out.println( "--- GroupMn Error 3 --- " + e);
}  }  } 
>>> 

\<consider start of num\><<<
if( qName.equals( "mn" ) ){
   inMn = true; level = 0;
   Attributes att = new AttributesImpl();
   super.startElement(ns, "mn-group", "mn-group", att); 
   this.ns = ns;
} else if( qName.equals( "mo" ) ){
   String cls = attr.getValue( "class" );
   if( (cls != null) && cls.equals("MathClass-punc") ){
      inMn = true; level = 0;
      Attributes att = new AttributesImpl();
      super.startElement(ns, "mn-group", "mn-group", att); 
      this.ns = ns;
}  }
>>>

\<consider end of num\><<<
if( !qName.equals( "mn" ) ){
  if( qName.equals( "mo" ) ){
     String cls = attr.getValue( "class" );
     if( (cls == null) || !cls.equals("MathClass-punc") ){
        inMn = false;  
        super.endElement(ns, "mn-group", "mn-group"); 
     }
  } else {
     inMn = false;
     super.endElement(ns, "mn-group", "mn-group"); 
} }
>>>

\<endElement: consider end of num\><<<
if( inMn ){
   inMn = false;
   super.endElement(ns, "mn-group", "mn-group"); 
}
>>>

%%%%%%%%%%%%%
\subsection{Outline of Post Original Partition}
%%%%%%%%%%%%%


\<numbers in math\><<<
<script element="mn-group" >
   `<missed partitions to mn groups`>
   <set name="merge" >
      `<open xslt script`>
      `<merge digits`>
      `<close xslt script`>
   </set>
   <xslt name="." xml="." xsl="merge" />
</script> 
>>>



\<missed partitions to mn groups\><<<
<dom name="." xml="." method="mnGroup" class="tex4ht.HtJsml" />
`<remove xml declaration`>
>>>


\<static void mnGroup(dom)\><<<
private static Document dom;
public static void mnGroup(Node d) {
  dom = (Document) d;
  setMnGroup(dom.getFirstChild());
}
private static void setMnGroup(Node node) {
  if( node.getNodeName().equals( "mn-group" ) ){
         boolean bool = false;
    `<get mn characters`>
    if( bool ){
      `<mark remove commas, if comma after period`>
      `<mark remove commas, if not spaced correctly`>
      `<cond remove commas`>
      `<remove punc, on consecutive periods`>
      `<remove punc at end`>
    }
    `<reset mn-group`>
  } else if (node.hasChildNodes()) {
     NodeList children = node.getChildNodes();
     for (int i = 0; i < children.getLength(); i++) {
        Node child = children.item(i);
        if (child.getNodeType() == Node.ELEMENT_NODE) {
           setMnGroup(child);
} }  }  }
>>>

%%%%%%%%%%%%%
\subsection{Mark Partition of MN Group}
%%%%%%%%%%%%%

Consecutive digits and punctation characters are grouped together, and
then ungrouped at characters that are marked by `x'.   


\<get mn characters\><<<
NodeList children = node.getChildNodes();
int n = children.getLength();
char [] digit = new char[n];
for (int i = 0; i < n; i++) {
  Node child = children.item(i).getFirstChild();
  if( child == null ){ 
     digit[i] = 'x';
  } else if( child.getNodeType() != Node.TEXT_NODE ){
     digit[i] = 'x';
  } else {
     String s = child.getNodeValue();
     if( s.length() != 1 ){ 
       digit[i] = 'x';
     } else {
       char ch = s.charAt(0);
       if(      (ch >= '0') && (ch <= '9') ){
          digit[i] = '0'; bool = true;
       } else if( (ch == '.') || (ch == ',') ){ digit[i] = ch; }
       else                                 { digit[i] = 'x';}
} }  }
>>>

\<mark remove commas, if comma after period\><<<
bool = false;
for (int i = 0; i < n; i++) {
  if( digit[i] == '.' ){ 
     for (; i < n; i++) {
       if( digit[i] == ',' ){ 
          bool = true; break;
     } }
     break; 
} }
>>>

\<mark remove commas, if not spaced correctly\><<<
if( !bool ){
  for (int i = 0; i < n; i++) {
    if( digit[i] == ',' ){
       if( ( ((i+3) >= n) 
             || (digit[i+1] != '0') 
             || (digit[i+2] != '0')
             || (digit[i+3] != '0') 
           )
           ||
           (
             ((i+4) < n) && (digit[i+4] == '0')
           )      
           ||
           (
             (i>3) && (digit[i-4] == '0')
           )      
       ){  bool = true; break;
       } else { i += 3; }
} } }
>>>

\<cond remove commas\><<<
if( bool ){
  for (int i = 0; i < n; i++) {
    if( digit[i] == ',' ){ digit[i] = 'x'; }
} }
>>>


\<remove punc, on consecutive periods\><<<
bool = false;
for (int i = 0; i < n; i++) {
  if( (digit[i] == 'x') 
      || (digit[i] == ',') ){ bool = false; }
  else if( digit[i] == '.' ){
    if( bool ){
       for (int j = 0; j < n; j++) {
         if( (digit[j] == '.') || (digit[j] == ',') ){
            digit[j] = 'x';
       } }
       break;
    } 
    bool = true;
} }
>>>

\<remove punc at end\><<<
if( digit[n-1] == '.' ){ digit[n-1] = 'x'; }
>>>

%%%%%%%%%%%%%
\subsection{Realize the Sub Partition}
%%%%%%%%%%%%%


\<reset mn-group\><<<
Node parent = node.getParentNode();
Element g = dom.createElement( "mn-group-s" );
Element cur = dom.createElement( "mn-group" );
for(int i=0;  i<n; i++ ){
   Node child = node.getFirstChild();   
   node.removeChild( child );
   if( digit[i] == 'x' ){
      if( cur.hasChildNodes() ){ g.appendChild( cur ); }      
      g.appendChild( child );
      cur = dom.createElement( "mn-group" );
   } else {
      cur.appendChild( child );
}  }  
if( cur.hasChildNodes() ){ g.appendChild( cur ); }
parent.replaceChild( g, node );
>>>


%%%%%%%%%%%%%
\subsection{Reset MNs}
%%%%%%%%%%%%%




\<merge digits\><<<
<xsl:template match="mn-group" >
   <xsl:choose>
      <xsl:when test=" not(child::mn) ">
        <xsl:apply-templates select="*|text()" />
      </xsl:when>
      <xsl:otherwise>
         <mn>
           <xsl:value-of select="." />
         </mn>
      </xsl:otherwise>
   </xsl:choose>
</xsl:template> 
>>>








\<remove super mn-group\><<<
<xsl:template match="mn-group-s" >
  <xsl:apply-templates select="*|@*|text()|comment()" />
</xsl:template> 
>>>



%%%%%%%%%%%%%
\subsection{Pronounced Letters}
%%%%%%%%%%%%%



\<sayas math letters \><<<
<xsl:template match="mi[ .='a' ]" >
  <xsl:copy>
    <xsl:apply-templates select="@*" />
    <SAYAS SUB="aih">a</SAYAS>   
  </xsl:copy>
</xsl:template> 
>>>


%%%%%%%%%%%%%
\subsection{Pronounced Punctuation in math}
%%%%%%%%%%%%%

TTSs might ignore punctuation marks. We want these
symbols in math.

\<sayas punctuation\><<<
<xsl:template match="mo[ 
    (@class='MathClass-punc')
    and ( (.='.') or (.=',')  or (.='!') 
          or (.=';') or (.=':') or (.='?') )
]" >
  <xsl:copy>
    <xsl:apply-templates select="@*" />
    <BREAK SIZE="small"/>
    <xsl:choose>
       <xsl:when test=" .='.' " >
          <SAYAS SUB="dot">.</SAYAS>
       </xsl:when>
       <xsl:when test=" .=',' " >
          <SAYAS SUB="comma">,</SAYAS>
       </xsl:when>
       <xsl:when test=" .=';' " >
          <SAYAS SUB="semicolon">;</SAYAS>
       </xsl:when>
       <xsl:when test=" .=':' " >
          <SAYAS SUB="colon">:</SAYAS>
       </xsl:when>
       <xsl:when test=" .='!' " >
          <SAYAS SUB="exclamation mark">!</SAYAS>
       </xsl:when>
       <xsl:when test=" .='?' " >
          <SAYAS SUB="question mark">?</SAYAS>
       </xsl:when>
    </xsl:choose>
    <BREAK SIZE="small"/>
  </xsl:copy>
</xsl:template> 
>>>


\<sayas digits\><<<
<xsl:template match="mn[
   `<mn modified above`>
   or
   `<mn modified under`>
]" >
  <xsl:copy>
    <SAYAS CLASS="digits">
      <xsl:value-of select="." />
    </SAYAS>
  </xsl:copy>
</xsl:template> 
>>>


\<mn modified above\><<<
ancestor::mover[
  not( descendant::*[
        (count( `<math content element`> ) &gt; 1)
     ])
  and
  (
     preceding-sibling::*[1][self::dot or self::mn-group-s]
     or
     following-sibling::*[1][self::dot or self::mn-group-s]
  )
] 
>>>



\<mn modified under\><<<
ancestor::munder[
  not( descendant::*[
        (count( `<math content element`> ) &gt; 1)
     ])
  and
  (
     preceding-sibling::*[1][self::dot or self::mn-group-s]
     or
     following-sibling::*[1][self::dot or self::mn-group-s]
  )
] 
>>>


%%%%%%%%%%%%%%%%%%
\section{Process Characters}
%%%%%%%%%%%%%%%%%%





%%%%%%%%%%%%%
\subsection{'minus' into 'negative'}
%%%%%%%%%%%%%



\<'minus' into 'negative'\><<<
<xsl:template match="mo[
    (@class = 'MathClass-bin')
  and
    ( normalize-space(.) =
      normalize-space(descendant::span[@class = 'ch 2212'])
    )
  and 
    (   preceding-sibling::*[1][
             (@title='speech-extra')
             or (@class='MathClass-bin')
             or (@class='MathClass-rel')  
             or (@class='MathClass-punc')  
             or (@class='MathClass-op')  
             or self::mn-group-s [ child::* [
                   (position() = last())
                  and
                   (@class='MathClass-punc')  
        ]       ] ]
        and
        (
           following-sibling::*[1][ self::mn-group-s 
                                 or self::mi ]
          or
           (count(following-sibling::*[
               not(@title = 'speech-extra')
            ]) = 1)
        )
      or
        not(preceding-sibling::*) and following-sibling::*
    )
]" >
  <xsl:copy>
    <xsl:attribute name="class">
       <xsl:text>mo-unary</xsl:text>
    </xsl:attribute>
    <xsl:apply-templates select="*|text()|comment()" 
                         mode="minus-neg" />
  </xsl:copy>
</xsl:template> 
>>>


\<'minus' into 'negative'\><<<
<xsl:template match="*|@*|text()|comment()" mode="minus-neg" >
  <xsl:copy>
    <xsl:apply-templates select="*|@*|text()|comment()" 
                         mode="minus-neg" />
  </xsl:copy>
</xsl:template> 
>>>

\begin{verbatim}
<span class="begin-script"> subscript </span>
<mo class="MathClass-bin">
  <span class="char">
    <span class="ch 2212">minus</span>
  </span>
</mo>
<mn>2</mn>
\end{verbatim}




\<'minus' into 'negative'\><<<
<xsl:template match="text()" mode="minus-neg" >
  <xsl:choose>
    <xsl:when test=" . = 'minus' " >
      <xsl:text>negative</xsl:text>
    </xsl:when>
    <xsl:otherwise>
      <xsl:value-of select="." />
    </xsl:otherwise>
  </xsl:choose>
</xsl:template> 
>>>


%%%%%%%%%%%%%%%%%%
\subsection{Capital Letters in Math}
%%%%%%%%%%%%%%%%%%

\<capital math letters\><<<
<script element="mi" >
  <set name="math-cap" >
     `<open xslt script`>
     `<span.mi cap letters`> 
     `<close xslt script`>
  </set>
  <xslt name="." xml="." xsl="math-cap" />
</script> 
>>>


\<span.mi cap letters\><<<
<xsl:template match="mi" >
  <xsl:copy>
      <xsl:apply-templates select="@*" />
      <xsl:if test=" string-length(.) = 1 " >      
         <xsl:if test="
                     (translate(.,'ABCDEFGHIJKLMNOPQRSTUVWXYZ',
                                  '') = '' )
                  " >      
            <span class="capital-description" 
                  title="speech-extra" >
               <xsl:text> capital </xsl:text>
            </span>
         </xsl:if>      
      </xsl:if>      
      <xsl:apply-templates select="*|text()|comment()" />
  </xsl:copy>
</xsl:template> 
>>>



%%%%%%%%%%%%%
\subsection{Right Apostrophes}
%%%%%%%%%%%%%



\<replace characters\><<<
<script element="PARA" >
  <set name="apostro" >
     `<open xslt script`>
     `<right apostrophe in prose`>
     `<delimiters in marks of enumerated lists`>
     `<remove empty paragraphs`>
     `<close xslt script`>
  </set>
  <xslt name="." xml="." xsl="apostro" />
</script> 
>>>


\<right apostrophe in prose\><<<
<xsl:template match="span[
     (@class = 'char-del') 
    and  child::span [ @class = 'ch 2019' ]
]" >
   <xsl:text>'</xsl:text>
</xsl:template> 
>>>


%%%%%%%%%%%%%
\subsection{Delimiters on List Marks}
%%%%%%%%%%%%%



\<delimiters in marks of enumerated lists\><<<
<xsl:template match="PROS[  parent::SENT[@class='ol-mark'] ]" >
   <xsl:copy>
      <xsl:apply-templates 
           select="*[not(@class='char-del')]|@*|text()|comment()" />  
   </xsl:copy>
</xsl:template> 
>>>


%%%%%%%%%%%%%%%%%%
\section{Clean Up}
%%%%%%%%%%%%%%%%%%


%%%%%%%%%%%%%
\subsection{Extra Math Pauses}
%%%%%%%%%%%%%


Deletes pauses that are not separated by text from their immediate
predecessors. 

An earlier XSLT-based approach was very slow, and died
on nodes with large numbers of children.


\<eliminate extra math pauses\><<<
<sax name="." xml="." content-handler="tex4ht.JsmlMathBreak" />
>>>


\AtEndDocument{ 
   \OutputCodE\<JsmlMathBreak.java\> 
} 
\Needs{"
    javac -classpath /home/4/gurari/tex4ht.dir/texmf/tex4ht/bin/tex4ht.jar
          JsmlMathBreak.java
          -d /home/4/gurari/xtpipes.dir/. 
"} 
 
\<JsmlMathBreak.java\><<< 
package tex4ht;
/* JsmlMathBreak.java (`version), generated from `jobname.tex
   Copyright (C) 2009-2010 TeX Users Group
   Copyright (C) `CopyYear.2002. Eitan M. Gurari
`<TeX4ht copyright`> */

import xtpipes.XtpipesUni;
import org.xml.sax.*;
import org.xml.sax.helpers.DefaultHandler;
import java.io.*;
import java.lang.reflect.*;
import java.util.HashMap;

public class JsmlMathBreak extends DefaultHandler {
        PrintWriter out = null;
        boolean delete = false;
  public JsmlMathBreak(PrintWriter out, 
                       HashMap<String,Object> scripts,
                       Method method, PrintWriter log, boolean trace) {
    this.out = out;
  }
  public void characters(char[] ch, int start, int length) {
    String s = XtpipesUni.toUni(ch, start, length, "<>&");
    out.print( s );
    if( !s.trim().equals("") ){
       delete = false;
  } }
  public void startElement(String ns, String sName,
                                      String qName,
                                      Attributes atts) {
    if( !( delete && qName.equals("BREAK") ) ){
      String s = "<" + qName + "\n";
      for (int i = 0; i < atts.getLength(); i++) {
        String name = atts.getQName(i);
        if (name != "xmlns") {
          s += (" " + name + "=\"" 
              + XtpipesUni.toUni(atts.getValue(i), "<>&\"")
              + "\"");
      } }
      if( qName.equals( "BREAK" ) ){
        s += "/";
        delete = true;
      }
      s += ">";
      out.print(s);
  } }
  public void endElement(String ns, String sName, String qName) {
    if( !qName.equals( "BREAK" ) ){
      String s = "</" + qName + ">";
      out.print(s);
} } }
>>>



\begin{verbatim}
  </mrow> 
  <span class="end-root" title="speech-extra"> 
    <BREAK SIZE="medium"/> 
    <level prefix="end" depth="1">end end root</level> 
    <BREAK SIZE="medium"/> 
  </span> 
</mroot> 
<span class="end-math" title="speech-extra"> 
  <BREAK SIZE="small"/> 
  <PROS PITCH="-5">end math</PROS> 
  <BREAK SIZE="medium"/> 
\end{verbatim}


%%%%%%%%%%%%%
\subsection{Line Breaks}
%%%%%%%%%%%%%


\<br into BREAK\><<<
<script element="br" >
  <set name="br" >
     `<open xslt script`>
     `<append br with BREAK`>
     `<close xslt script`>
  </set>
  <xslt name="." xml="." xsl="br" />
</script> 
>>>


\<append br with BREAK\><<<
<xsl:template match="br" >
   <xsl:copy>
       <xsl:apply-templates select="@*" />
   </xsl:copy>
   <BREAK SIZE="small"/> 
</xsl:template> 
>>>
>>>


%%%%%%%%%%%%%%%%%%
\subsection{Empty Elements}
%%%%%%%%%%%%%%%%%%



\marginpar{??????}

Can we avoid empty non-empty representations without of the
following(Example: 
\verb+\documentclass{article} \begin{document} \section{Section One} This is Section 1 \end{document} +
)


\<short notation for empty elements\><<<
<script element="BREAK" >
  <set name="BREAK" >
     `<open xslt script`>
     `<empty BREAK`>
     `<close xslt script`>
  </set>
  <xslt name="." xml="." xsl="BREAK" />
</script> 
>>>


\<empty BREAK\><<<
<xsl:template match="BREAK" >
   <xsl:copy>
       <xsl:apply-templates select="@*" />
   </xsl:copy>
</xsl:template> 
>>>



%%%%%%%%%%%%%
\subsection{Empty Array Cells}
%%%%%%%%%%%%%

Empty array celles at end of rows can make it more difficult to 
detect deletable baseline indicators.



\<remove multline eqnum cell\><<<
<script element="tr" >
  <set name="eqnum" >
     `<open xslt script`>
     `<remove empty cells from rows`> 
     `<close xslt script`>
  </set>
  <xslt name="." xml="." xsl="eqnum" />
</script> 
>>>

\<remove multline eqnum cell\><<<
<script element="mtr" >
  <set name="eqnum" >
     `<open xslt script`>
     `<remove empty cells from rows`> 
     `<close xslt script`>
  </set>
  <xslt name="." xml="." xsl="eqnum" />
</script> 
>>>

\<remove empty cells from rows\><<<
<xsl:template match="div[ 
    (parent::tr or parent::mtr)
    and ( normalize-space(.) = '' )
    and not( normalize-space(following-sibling::*) != '' )
]" >
</xsl:template> 
>>>


%%%%%%%%%%%%%
\subsection{Paragraphs (JsmlFilter.java)}
%%%%%%%%%%%%%

\<remove empty paragraphs\><<<
<xsl:template match=" PARA[ normalize-space(.) = '' ] " >
</xsl:template> 
>>>


The following adds PARA on `p' paragraphs, ul.




\AtEndDocument{ 
   \OutputCodE\<JsmlFilter.java\> 
} 
\Needs{"
    javac JsmlFilter.java -d /home/4/gurari/xtpipes.dir/.  
"} 
 
\<JsmlFilter.java\><<< 
package tex4ht;
/* JsmlFilter.java (`version), generated from `jobname.tex
   Copyright (C) 2009-2010 TeX Users Group
   Copyright (C) `CopyYear.2002. Eitan M. Gurari
`<TeX4ht copyright`> */
import org.xml.sax.helpers.*; 
import org.xml.sax.*; 
import java.io.PrintWriter; 
 
public class JsmlFilter extends XMLFilterImpl { 
     PrintWriter out = null; 
   public JsmlFilter( PrintWriter out, PrintWriter log, boolean trace ){ 
     this.out = out; 
   } 
   public void startElement(String ns, String sName, 
                           String qName, Attributes attr) { 
      try{ 
        if( `<elements for PARA?`> ){
          Attributes att = new AttributesImpl();
          super.startElement(ns, "PARA", "PARA", att); 
        }
        super.startElement(ns, sName, qName, attr); 
      } catch( Exception e ){
        System.out.println( "--- JsmlFilter Error 1 --- " + e);
   }  } 
   public void endElement(String ns, String sName, String qName){ 
      try{
        super.endElement(ns, sName, qName);  
        if( `<elements for PARA?`> ){
             super.endElement(ns, "PARA", "PARA"); 
        }
      } catch( Exception e ){
        System.out.println( "--- JsmlFilter Error 2 --- " + e);
}  }  } 
>>> 




\<elements for PARA?\><<<
   qName.equals( "p" ) 
|| qName.equals( "h2" ) 
|| qName.equals( "h3" ) 
|| qName.equals( "h4" ) 
|| qName.equals( "ul" ) 
|| qName.equals( "ol" ) 
|| qName.equals( "li" ) 
|| qName.equals( "dd" ) 
|| qName.equals( "dl" )
>>>


%%%%%%%%%%%%%
\subsection{Remove Split Arrays}
%%%%%%%%%%%%%

\<remove empty split entries\><<<
<script element="div::split-side" >
  <set name="clean-split" >
     `<open xslt script`>
     `<get content template`>
     `<clean math split`> 
     `<close xslt script`>
  </set>
  <xslt name="." xml="." xsl="clean-split" />
</script> 
>>>


\<clean math split\><<<
<xsl:template match=" div[ @class='split-side' ] 
" >
    <xsl:variable name="content">
       <xsl:apply-templates select="*" mode="content" />
    </xsl:variable>
    <xsl:if test=" 
       string-length( normalize-space( $content )) != 0
    " >       
       <xsl:copy>
          <xsl:apply-templates select=" *|@*|text()|comment() " />
       </xsl:copy>
    </xsl:if>
</xsl:template> 
>>>


%%%%%%%%%%%%%
\section{Eliminate Inline Math Narrative}
%%%%%%%%%%%%%


%%%%%%%%%%%%%
\subsection{Core Content of  One Element}
%%%%%%%%%%%%%


\<eliminate inline math narrative NO\><<<
<xsl:template match="span[
    (@class = 'inline-math')
  and
    (count( `<math content element`> ) = 1)
]" >
  <xsl:copy>
    <xsl:choose>
       <xsl:when test="child::mfrac or child::msqrt
                    or child::mover or child::munder
       ">
          <xsl:attribute name="class">
             <xsl:text>semi-math</xsl:text>
          </xsl:attribute>
          <BREAK SIZE="small"/>
          <xsl:apply-templates
              select="`<math content element`>" />
       </xsl:when>
       <xsl:otherwise>
          `<non-adjacent narrative`>
       </xsl:otherwise>
    </xsl:choose>
  </xsl:copy>
</xsl:template> 
>>>



\<non-adjacent narrative\><<<
<xsl:variable name="content">
   <xsl:apply-templates
       select="`<math content element`>"
         mode="content" />
</xsl:variable>
<xsl:choose>
   `<math narrative of length one`>
   `<math narrative of just one element`>
   `<simple sub or sup`>
   <xsl:otherwise>
      <xsl:apply-templates
                    select="*|@*|text()|comment()" />
   </xsl:otherwise>
</xsl:choose>
>>>





\<math content element\><<<
child::*[ not(@title = 'speech-extra') 
          and not( self::BREAK )
        ]
>>>




\<math narrative of length one\><<<
<xsl:when test=" 
   string-length( normalize-space( $content )) = 1
" >
   <xsl:attribute name="class">
      <xsl:text>semi-math</xsl:text>
   </xsl:attribute>
   <xsl:apply-templates
              select="`<math content element`>" />
</xsl:when>
>>>

\<math narrative of just one element\><<<
<xsl:when test=" 
   not(`<math content element`> 
        / descendant::* [
            count(child::*[ 
                not(self::BREAK) 
            ] ) &gt; 1
          ] )
" >
   <xsl:attribute name="class">
      <xsl:text>semi-math</xsl:text>
   </xsl:attribute>
   <PROS PITCH="-5">
   <BREAK SIZE="small"/>
      <xsl:apply-templates
              select="`<math content element`>" />
   <BREAK SIZE="small"/>
   </PROS> 
</xsl:when>
>>>


\<simple sub or sup\><<<
<xsl:when test="  child::*[
     (position() = 2)
   and
     (self::msub or self::msup or self::msubsup)
]" >
    <xsl:variable name="content">
       <xsl:apply-templates
           select="child::*[2] / child::*[
                   (@class = 'mrow-base')
                or (@class = 'limits-mrow-base') ]" 
             mode="content" />
    </xsl:variable>
    <xsl:choose>
       <xsl:when test=" 
          string-length( normalize-space( $content )) = 1
         or
          (translate($content,'0123456789 ','') = '')          
       " >
          <xsl:attribute name="class">
             <xsl:text>semi-math</xsl:text>
          </xsl:attribute>
          <PROS PITCH="-5">
          <BREAK SIZE="small"/>
             <xsl:apply-templates select="*[2]" />
          <BREAK SIZE="small"/>
          </PROS> 
      </xsl:when>
       <xsl:otherwise>
          <xsl:apply-templates select="*|@*|text()" />
       </xsl:otherwise>
    </xsl:choose>
</xsl:when>
>>>


\begin{verbatim}
<span class="inline-math">
  <msub>
    <mrow class="mrow-base">
      <mi>C</mi>
    </mrow>
    <mrow class="mrow-sub">
      <mi>i</mi>
    </mrow>
  </msub>
</span>
\end{verbatim}




%%%%%%%%%%%%%
\subsection{Unary Op}
%%%%%%%%%%%%%

The `mo-unary' is established in the first pass over inline-math,
so we need to wait for the second pass with the following.


\<eliminate inline math narrative NO 2\><<<
<xsl:template match="span[
    (@class = 'inline-math')
  and
    (count( `<math content element`> ) = 2)
  and
    (  child::mo[ 
          (@class = 'mo-unary')
        and
          following-sibling::*[1]
          / self::mi
       ]    
      or
       child::mi[
          following-sibling::*[1]
          [ (@class = 'MathClass-open-close')
            and
            not(child::*[2] 
                / child::*[ not(self::BREAK) ]
                / child::*[ not(self::BREAK) ] )
       ]  ]
    )
]" >
  <xsl:copy>
    <xsl:choose>
       `<when func on shallow arg`>
       <xsl:otherwise>
          `<unary op on short content`>
       </xsl:otherwise>
    </xsl:choose>
  </xsl:copy>
</xsl:template> 
>>>

\<when func on shallow arg\><<<
<xsl:when test=" 
   child::mrow[ @class = 'MathClass-open-close' ]
" >
   <xsl:attribute name="class">
      <xsl:text>semi-math</xsl:text>
   </xsl:attribute>
   <PROS PITCH="-5">
   <BREAK SIZE="small"/>
      <xsl:apply-templates
              select="`<math content element`>" />
   <BREAK SIZE="small"/>
   </PROS> 
</xsl:when>
>>>

\<unary op on short content\><<<
<xsl:attribute name="class">
   <xsl:text>semi-math</xsl:text>
</xsl:attribute>
<PROS PITCH="-5">
<BREAK SIZE="small"/>
   <xsl:apply-templates
           select="`<math content element`>" />
<BREAK SIZE="small"/>
</PROS> 
>>>



%%%%%%%%%%%%%
\subsection{Shallow Expression}
%%%%%%%%%%%%%


\<eliminate inline math narrative NO 2\><<<
<xsl:template match="span[
    (@class = 'inline-math')
  and
     not( child::*[
         not(self::mo) 
         and not(self::mi) 
         and not(self::mn) 
         and not(self::mn-group-s) 
         and not( @title='speech-extra' ) 
         and not( self::mfrac[  
            preceding-sibling::*[1][self::mn ]
           ]
         )
         and not( self::msub
                  / child::*[1][
                       child::mo[ @class = 'MathClass-op' ]
                       and
                       not(child::*[2])
                    ]
         )
     ]  )
]" >
  <xsl:copy>
    `<shallow expression`>
  </xsl:copy>
</xsl:template> 
>>>

\<shallow expression\><<<
<xsl:attribute name="class">
   <xsl:text>semi-math</xsl:text>
</xsl:attribute>
<PROS PITCH="-5">
<BREAK SIZE="small"/>
   <xsl:apply-templates
           select="`<math content element`> | text()" />
<BREAK SIZE="small"/>
</PROS> 
>>>




%%%%%%%%%%%%%
\subsection{Shallow Delimited Expression}
%%%%%%%%%%%%%


\<eliminate inline math narrative NO 2\><<<
<xsl:template match="span[
    (@class = 'inline-math')
  and
    (count( `<math content element`> ) = 1)
  and
    child::mrow[ @class='MathClass-open-close' ]
    / child::mrow[ 
       not( child::*[
         not(self::mo) and not(self::mn-group-s) 
         and not(self::mi) and not(self::BREAK) 
         and not( @title='speech-extra' ) 
       ])
    ]
]" >
  <xsl:copy>
    `<shallow delimited expression`>
  </xsl:copy>
</xsl:template> 
>>>

\<shallow delimited expression\><<<
<xsl:attribute name="class">
   <xsl:text>semi-math</xsl:text>
</xsl:attribute>
<PROS PITCH="-5">
<BREAK SIZE="small"/>
   <xsl:apply-templates select="mrow" />
<BREAK SIZE="small"/>
</PROS> 
>>>



%%%%%%%%%%%%%%%%%%
\section{Superscripts and Subscripts}
%%%%%%%%%%%%%%%%%%

%%%%%%%%%%%%%
\subsection{Insert Sub-Levels Info}
%%%%%%%%%%%%%




\<set levels for sub and sup scripts\><<<
<dom name="." xml="." method="scriptLevel" class="tex4ht.HtJsml" />
`<remove xml declaration`>
>>>



\<static void scriptLevel(dom)\><<<
public static void scriptLevel(Node dom) {
   setScriptLevel(dom.getFirstChild(), "");
}
private static void setScriptLevel(Node node, String prefix) {
  String clName = null;
  if (node.hasChildNodes()) {
    if (node.hasAttributes()) {
      Node cl = node.getAttributes().getNamedItem("class");
      if (cl != null) {
        clName = cl.getNodeValue();
        if( clName.equals("mrow-sub")
            ||
            clName.equals("mrow-super")
        ){
          `<append script prefix`>
    } } }
    String ndName = node.getNodeName();
    if( 
      ndName.equals("msqrt")
      ||
      ndName.equals("mroot")
    ){  prefix = ""; }
    `<script invoke children`>
} }
>>>

\<script invoke children\><<<
NodeList children = node.getChildNodes();
for (int i = 0; i < children.getLength(); i++) {
   Node child = children.item(i);
   if (child.getNodeType() == Node.ELEMENT_NODE) {
      setScriptLevel(child, prefix);
}  }
>>>

\<append script prefix\><<<
if( !prefix.equals("") ){
  `<use current script prefix`>
}
if( clName.equals( "mrow-sub" ) ){ 
    prefix += " sub "; 
} else if( clName.equals( "mrow-super" ) ){ 
    prefix += " super "; 
}
>>>


\<use current script prefix\><<<
Node child = node.getFirstChild();
if( (child.getNodeType() == Node.ELEMENT_NODE) 
    &&
    child.hasAttributes()
){
   Node cls = child.getAttributes().getNamedItem("class");
   if (cls != null) {
      String clsName = cls.getNodeValue();
      if ( clsName.equals("begin-script")
           ||
           clsName.equals("mid-script")
      ) {
         child = child.getFirstChild();
         String s = child.getNodeValue();
         ((org.w3c.dom.Text) child).setData( prefix + s );
}  }  }
>>>
 


%%%%%%%%%%%%%
\subsection{Eliminate  End Script Marks}
%%%%%%%%%%%%%


The following takes care of end-scripts that semantically can be
merged into other end markers.

\<eliminate baseline script marks\><<<
<xsl:template match="span[
   (@class = 'end-script')
   and
   ancestor::* [ following-sibling::* [
                         not( @class = 'content-less' )
                      ] 
               ][1]
           / following-sibling::* [
                         not( @class = 'content-less' )
                     ][1] 
           / self::*
    [
       (@class = 'end-math') 
       or
       (@class = 'end-script') 
       or
       (@class = 'end-root') 
       or
       (@class = 'end-stack') 
       or
       (@class = 'mid-stack') 
       or
       (@class = 'end-array') 
       or
        self::td or self::mtd or self::tr or self::mtr 
       or
       (@title = 'implicit-baseline') 
    ]
]"  >
  <BREAK SIZE="small"/>
</xsl:template> 
>>>






The following deals with endscripts before left sides of
tensors.



\<eliminate baseline script marks\><<<
<xsl:template match="span[
   (@class = 'end-script')
   and
    parent::mrow / parent::msub
   and
   ancestor::* [ following-sibling::* ][1]
   / following-sibling::* [ normalize-space(.) != '' ][1]
    [
       (self::msub or self::msubsup)
     and
       (normalize-space(child::mrow[ 
          @class = 'mrow-base' ]) = '')
    ]
]"  >
  <BREAK SIZE="small"/>
</xsl:template> 
>>>



\<eliminate baseline script marks\><<<
<xsl:template match="span[
   (@class = 'end-script')
   and
     parent::mrow /parent::*[ self::msup or self::msubsup]
   and
   ancestor::* [ following-sibling::* ][1]
   / following-sibling::* [ normalize-space(.) != '' ][1]
    [
       self::msup 
     and
       (normalize-space(child::mrow[ 
          @class = 'mrow-base' ]) = '')
    ]
]"  >
  <BREAK SIZE="small"/>
</xsl:template> 
>>>








\begin{verbatim}
    <span class="end-script" title="speech-extra"> 
      <PROS PITCH="-5"> 
        <BREAK SIZE="medium"/> 
        <span class="scripts-extra"> baseline </span> 
        <BREAK SIZE="small"/> 
      </PROS> 
    </span> 
  </mrow> 
</msup> 
<span class="tiny-space"/> 
<msup> 
  <mrow class="mrow-base"/> 
  <mrow class="mrow-super"> 

\end{verbatim}

%%%%%%%%%%%%%
\subsection{Clean Script Marks}
%%%%%%%%%%%%%


\<clean begin and mid script marks\><<<
<xsl:template match="span[
   (@class = 'begin-script')
   or (@class = 'mid-script')      
]" >
  <xsl:copy>
     <xsl:apply-templates select="@*" />
     <xsl:apply-templates select="*|text()|comment()" 
                          mode="clean-script" />  
  </xsl:copy>
</xsl:template> 
<xsl:template match="*"  mode="clean-script" >
  <xsl:copy>
     <xsl:apply-templates select="@*" />
     <xsl:apply-templates select="*|text()|comment()" 
                          mode="clean-script" />  
  </xsl:copy>
</xsl:template>   
<xsl:template match="text()"  mode="clean-script" >
</xsl:template>   
<xsl:template match="span[@class = 'scripts-extra']"
               mode="clean-script" >
  <xsl:copy>
     <xsl:apply-templates select="@*" />
     <xsl:value-of select=" 
        ancestor::span[
          (@class = 'begin-script')
           or (@class = 'mid-script') ] [1]
      " />
  </xsl:copy>
</xsl:template>   
>>>

The following is for script annotation on empty msu::base.

\<remove marks on empty scripts\><<<
<xsl:template match="span[
     ((@class = 'begin-script') or (@class = 'mid-script'))
   and
     following-sibling::*
     / following-sibling::span[ @class = 'end-script' ]
   and
     following-sibling::*[1]
     / descendant-or-self::*[ not(self::PROS) ][1]
     / child::*[1][
           (@class = 'mrow-base')
         and
           (normalize-space(.)='')
       ]
]" >
</xsl:template>   
>>>

\begin{verbatim}
<span class="begin-script" title="speech-extra"> 
  ...
</span> 
<PROS PITCH="+10"> 
  <msub> 
    <mrow class="mrow-base"/> 
    <mrow class="mrow-sub"> 
      ...
    </mrow> 
  </msub> 
  <mi>n</mi> 
</PROS> 
<span class="end-script" title="speech-extra"> 
  ...
\end{verbatim}



%%%%%%%%%%%%%
\subsection{Replace Nested Baseline Script Marks}
%%%%%%%%%%%%%


\<replace nested baseline script marks\><<<
<xsl:template match="span[
     (@class = 'end-script')
   and
     ancestor::*[ preceding-sibling::* [
       ((@class = 'begin-script') or (@class = 'mid-script')) ]]
]" >
  <xsl:copy>
     <xsl:apply-templates select="@*" />
     <xsl:apply-templates select="
            ancestor::*[ preceding-sibling::* [
                           ((@class = 'begin-script') or 
                            (@class = 'mid-script'))     ]][1]
            / preceding-sibling::* [
                           ((@class = 'begin-script') or 
                            (@class = 'mid-script'))  ][1]
      "  mode="script-copy" />
  </xsl:copy>
</xsl:template>   
>>>



\<replace nested baseline script marks\><<<
<xsl:template match="*|@*|text()|comment()"
               mode="script-copy" >
  <xsl:copy>
     <xsl:apply-templates select="*|@*|text()|comment()" />  
  </xsl:copy>
</xsl:template>   
<xsl:template match="span[
     (@class = 'begin-script') or 
     (@class = 'mid-script')       ]"  mode="script-copy" >
   <xsl:apply-templates select="*|text()|comment()" />  
</xsl:template>   
>>>


%%%%%%%%%%%%%
\subsection{Eliminate  Begin Script Marks}
%%%%%%%%%%%%%



\<eliminate begin script marks\><<<
<xsl:template match="span[
     (@class = 'begin-script')
   and
     following-sibling::* [1] / child::*[  
       (position() = 1)
       and 
       (@class = 'mrow-base')
       and
       ( normalize-space(.) = '' )
     ]
]"  >
</xsl:template> 
>>>



%%%%%%%%%%%%%
\subsection{Undo Empty Scripts}
%%%%%%%%%%%%%




\<special sub and super scripts\><<<
<script element="msup" >
   <set name="m-sub-sup" >
     `<open xslt script`>
     `<undo if empty su`> 
     `<close xslt script`>
   </set>
   <xslt name="." xml="." xsl="m-sub-sup" />
   `<superscript 2 and 3 into verbose`>
</script> 
<script element="msub" >
  `<sub script`>
</script> 
<script element="msubsup" >
  `<subsup prime`>
  `<sub sup script`>
  `<sub superscript 2 and 3 into verbose`>
</script> 
>>>

\<sub sup script\><<<
<set name="m-sub-sup" >
  `<open xslt script`>
  `<undo if empty su`> 
  `<close xslt script`>
</set>
<xslt name="." xml="." xsl="m-sub-sup" />
>>>



\<sub script\><<<
<set name="m-sub-sup" >
  `<open xslt script`>
  `<undo if empty su`> 
  `<msub of log`> 
  `<close xslt script`>
</set>
<xslt name="." xml="." xsl="m-sub-sup" />
>>>

\<undo if empty su\><<<
<xsl:template match="*[ 
    (self::msup or self::msub or self::msubsup)
  and
    not(
        child::mrow[ @class = 'mrow-sub' ] 
        / child::*[ not(@title = 'speech-extra')
                    and
                    (normalize-space(.) != '')
                  ]
    )
  and
    not(
        child::mrow[ @class = 'mrow-super' ] 
        / child::*[ not(@title = 'speech-extra')
                    and
                    (normalize-space(.) != '')
                  ]
    )
]" >
   <xsl:apply-templates select="child::mrow[ 
                          @class = 'mrow-base' ]/*" />
</xsl:template> 
>>>



\begin{verbatim}
<msup>
   <mrow class="mrow-base">
     .........
   </mrow>
   <wrow class="mrow-super">
     <span class="begin-script"> superscript </span>
     <span class="end-script"> baseline </span>
   </mrow>
</msup>
\end{verbatim}




%%%%%%%%%%%%%
\section{Special Subscripts and Superscripts}
%%%%%%%%%%%%%

%%%%%%%%%%%%%
\subsection{Squared}
%%%%%%%%%%%%%



\<superscript 2 and 3 into verbose\><<<
<set name="m-sup-2-3" >
  `<open xslt script`>
  `<superscript into squared and cube`> 
  `<close xslt script`>
</set>
<xslt name="." xml="." xsl="m-sup-2-3" />
>>>

\<superscript into squared and cube\><<<
<xsl:template match="msup[
    (normalize-space(
      child::mrow[ (@class = 'mrow-super') ] 
        / child::*[ not (@class = 'begin-script')
                    and
                    not (@class = 'end-script')
                  ]
     ) = '2')
  and `<su verbose not on op`>
  and (normalize-space(mrow[@class = 'mrow-base']) != '' )
]" >
   <xsl:copy>
      <xsl:apply-templates select="*|@*|text()|comment()" 
                           mode="squared" />
   </xsl:copy>
</xsl:template> 
>>>


\<su verbose not on op\><<<
not(
  child::mrow[ @class = 'mrow-base' ] 
    / child::span[ not( @title = 'speech-extra' ) ]
                 [ position() = last() ]
    / self::* [ @class = 'mo-op' ]    
)
>>>




\<superscript into squared and cube\><<<
<xsl:template match="*|@*|text()|comment()" 
                           mode="squared" >
   <xsl:copy>
     <xsl:choose>
        <xsl:when test=" @class = 'mrow-super' ">
           <xsl:apply-templates select="@*" />
           <mo class="mo-op">
             <xsl:text> squared </xsl:text>
           </mo>
       </xsl:when>  
        <xsl:otherwise>
           <xsl:apply-templates select="*|@*|text()" />
        </xsl:otherwise>
     </xsl:choose>
   </xsl:copy>
</xsl:template> 
>>>






\begin{verbatim}
<msup>
  <mrow class="mrow-base">
    <mo class="mo-op">cos</mo>
  </mrow>
  <mrow class="mrow-super">
    <span class="begin-script"> superscript </span>
    <span class="mn">2</span>
    <span class="end-script"> baseline </span>
  </mrow>
</msup>
\end{verbatim}




\<superscript into squared and cube\><<<
<xsl:template match="msup[
    (normalize-space(
      child::mrow[ (@class = 'mrow-super') ] 
        / child::*[ not (@class = 'begin-script')
                    and
                    not (@class = 'end-script')
                  ]
     ) = '3')
  and `<su verbose not on op`>
  and (normalize-space(mrow[@class = 'mrow-base']) != '' )
]" >
   <xsl:copy>
      <xsl:apply-templates select="*|@*|text()|comment()" 
                           mode="cube" />
   </xsl:copy>
</xsl:template> 
>>>



\<superscript into squared and cube\><<<
<xsl:template match="*|@*|text()|comment()" 
                           mode="cube" >
   <xsl:copy>
     <xsl:choose>
        <xsl:when test=" @class = 'mrow-super' ">
           <xsl:apply-templates select="@*" />
           <mo class="mo-op">
             <xsl:text> cube </xsl:text>
           </mo>
        </xsl:when>  
        <xsl:otherwise>
           <xsl:apply-templates select="*|@*|text()" />
        </xsl:otherwise>
     </xsl:choose>
   </xsl:copy>
</xsl:template> 
>>>


\<sub superscript 2 and 3 into verbose\><<<
<set name="m-subsup-2-3" >
  `<open xslt script`>
  `<subsup into sub squared and cube`> 
  `<close xslt script`>
</set>
<xslt name="." xml="." xsl="m-subsup-2-3" />
>>>





\<subsup into sub squared and cube\><<<
<xsl:template match="msubsup[
    (normalize-space(
      child::mrow[ (@class = 'mrow-super') ] 
        / child::span[ (@class != 'mid-script')
                       and
                       (@class != 'end-script')
                     ]
     ) = '2')
  and `<su verbose not on op`>
]" >
   <xsl:copy>
      <xsl:attribute name="class" >
         <xsl:text>msub</xsl:text>
      </xsl:attribute>
      <xsl:apply-templates select="*|text()|comment()" 
                           mode="sub-squared" />
   </xsl:copy>
</xsl:template> 
>>>



\<subsup into sub squared and cube\><<<
<xsl:template match="*|@*|text()|comment()" 
                           mode="sub-squared" >
   <xsl:copy>
     <xsl:choose>
        <xsl:when test=" @class = 'mrow-sub' ">
           <xsl:apply-templates select="*[
                       not( @class = 'end-script' )
                     ]
                           |@*|text()|comment()" />
           <xsl:apply-templates select="
               following-sibling::*[1] / *[
                        @class = 'end-script' 
                  ] " />
        </xsl:when>  
        <xsl:when test=" @class = 'mrow-super' ">
           <xsl:attribute name="class" >
               <xsl:text>squared-super</xsl:text>
           </xsl:attribute>
           <mo class="mo-op">
             <xsl:text> squared </xsl:text>
           </mo>
        </xsl:when>  
        <xsl:otherwise>
           <xsl:apply-templates select="*|@*|text()|comment()" />
        </xsl:otherwise>
     </xsl:choose>
   </xsl:copy>
</xsl:template> 
>>>


\<subsup into sub squared and cube\><<<
<xsl:template match="msubsup[
    (normalize-space(
      child::mrow[ (@class = 'mrow-super') ] 
        / child::span[ (@class != 'mid-script')
                       and
                       (@class != 'end-script')
                     ]
     ) = '3')
  and `<su verbose not on op`>
]" >
   <xsl:copy>
      <xsl:attribute name="class" >
         <xsl:text>msub</xsl:text>
      </xsl:attribute>
      <xsl:apply-templates select="*|text()|comment()" 
                           mode="sub-cube" />
   </xsl:copy>
</xsl:template> 
>>>



\<subsup into sub squared and cube\><<<
<xsl:template match="*|@*|text()|comment()" 
                           mode="sub-cube" >
   <xsl:copy>
     <xsl:choose>
        <xsl:when test=" @class = 'mrow-sub' ">
           <xsl:apply-templates select="*[
                       not( @class = 'end-script' )
                     ]
                           |@*|text()|comment()" />
           <xsl:apply-templates select="
               following-sibling::*[1] / *[
                        @class = 'end-script' 
                  ] " />
        </xsl:when>  
        <xsl:when test=" @class = 'mrow-super' ">
           <xsl:attribute name="class" >
               <xsl:text>cube-super</xsl:text>
           </xsl:attribute>
           <mo class="mo-op">
             <xsl:text> cube </xsl:text>
           </mo>
        </xsl:when>  
        <xsl:otherwise>
           <xsl:apply-templates select="*|@*|text()|comment()" />
        </xsl:otherwise>
     </xsl:choose>
   </xsl:copy>
</xsl:template> 
>>>



%%%%%%%%%%%%%
\subsection{Numeric Subscripts (Rule 77)}
%%%%%%%%%%%%%

\<compress numeric subscripts\><<<
  <set name="num-sub" >
     `<open xslt script`>
     `<num sub`> 
     `<close xslt script`>
  </set>
  <xslt name="." xml="." xsl="num-sub" />
>>>



\<num sub\><<<
<xsl:template match="msub[
  (
    `<num sub on non-primed base`>
    or
    `<num sub on primed base`>
  )
  and
    child::mrow[ @class = 'mrow-sub' ] 
    / child::*[ not(@title = 'speech-extra') ][1]
    / self::*[
          normalize-space(.) 
          = normalize-space( descendant::mn ) ]
]" >  
  <xsl:copy>
     <xsl:apply-templates select="@*" />     
     <BREAK SIZE="small"/>
     <xsl:apply-templates select="*[1]" />
     <mrow class="mrow-sub">
        <xsl:apply-templates
           select="*[2] / *[ 
                    not(@title = 'speech-extra')  
          ]" />
        <BREAK SIZE="small"/>
     </mrow>
  </xsl:copy>
</xsl:template> 
>>>

\<num sub on non-primed base\><<<
(count( child::mrow[ @class = 'mrow-base' ] 
           / child::* ) = 1 )
  and child::mrow[ @class = 'mrow-base' ] / descendant::mi
  and not( ancestor::*[
      self::msub or self::msup or self::msubsup
    ] )
>>>

\<num sub on primed base\><<<
(
  count( child::mrow[ @class = 'mrow-base' ] 
        / child::* [
          not(self::BREAK)
        ]
  ) = 2 )
  and child::mrow[ @class = 'mrow-base' ] [
         child::*[1][ self::mi ]
         and
         child::*[
           (position() &gt; 1)
           and
           (normalize-space(.) = 
            normalize-space(
               descendant-or-self::span[
                  (@class = 'ch 2032') or (@class = 'ch 2033') 
                                 or (@class = 'ch 2034') 
               ]
            ))
         ]
      ]
   and not( ancestor::*[
      self::msub or self::msup or self::msubsup
    ] )
>>>

\begin{verbatim}
<msub>
  <mrow class="mrow-base">
    <mi>x</mi>
  </mrow>
  <mrow class="mrow-sub">
    <span class="begin-script" title="speech-extra"> subscript </span>
    <mn>1</mn>
    <span class="end-script" title="speech-extra"> baseline </span>
  </mrow>
</msub>
\end{verbatim}



\<num sub\><<<
<xsl:template match="msubsup[
  (
    `<num sub on non-primed base`>
  )
  and
    child::mrow[ @class = 'mrow-sub' ] 
    / child::*[ not(@title = 'speech-extra') ][1]
    / self::*[
          normalize-space(.) 
          = normalize-space( descendant::mn ) ]
]" >  
  <msup>
     <BREAK SIZE="small"/>
     <xsl:apply-templates select="*[1]" />
     <mrow class="mrow-sub">
        <xsl:apply-templates
           select="*[2] / *[ 
                    not(@title = 'speech-extra')  
          ]" />
        <BREAK SIZE="small"/>
     </mrow>
     <xsl:apply-templates select="*[3]" />
  </msup>
</xsl:template> 
>>>

%%%%%%%%%%%%%
\subsection{Primes}
%%%%%%%%%%%%%

\<remove scrip indicators from primes\><<<
<xsl:template match="span[
    (@class = 'begin-script')
  and
    following-sibling::*[ not(@title = 'speech-extra') ] [1]
    / descendant-or-self::*[ not( self::PROS ) ][1]
    / 
         child::span / child::span[ 
            (@class = 'ch 2032') or (@class = 'ch 2033')
                                 or (@class = 'ch 2034')
    ]    
]" >
  <BREAK SIZE="small"/> 
</xsl:template> 
>>>


\<remove scrip indicators from primes\><<<
<xsl:template match="span[
    (@class = 'end-script')
  and
    preceding-sibling::*[ not(@title = 'speech-extra') ] [1]
    / descendant-or-self::*[ not( self::PROS ) ][1]
    / 
         child::span / child::span[ 
            (@class = 'ch 2032') or (@class = 'ch 2033')
                                 or (@class = 'ch 2034')
    ]    
]" >
  <BREAK SIZE="small"/> 
</xsl:template> 
>>>



\begin{verbatim}
<span class="msup">
  <span class="mrow-base">
    <span class="mi">x</span>
  </span>
  <span class="mrow-super">
    <span class="begin-script"> superscript </span>
    <span class="mo-op">
      <span class="char">
        <span class="ch 2033">double prime</span>
      </span>
    </span>
    <span class="end-script"> baseline </span>
  </span>
</span>
\end{verbatim}

\<subsup prime\><<<
<set name="subsup-prime" >
   `<open xslt script`>
   `<compress subsup prime`> 
   `<close xslt script`>
</set>
<xslt name="." xml="." xsl="subsup-prime" />
>>>


\<compress subsup prime\><<<
<xsl:template match="msubsup[ 
    (count(
        child::mrow [ @class='mrow-super' ]
       / child::* [ not(@title = 'speech-extra') ]
    ) = 1)
  and
    not(
        child::mrow [ @class='mrow-super' ]
       / child::* [ not(@title = 'speech-extra') ]
       / descendant::*[ preceding-sibling::* 
                        or following-sibling::*]
    )
  and
    child::mrow [ @class='mrow-super' ]
       / child::* [ not(@title = 'speech-extra') ] 
       / descendant-or-self::span[ @class='char' ]
       / child::span[
            (@class = 'ch 2032') or (@class = 'ch 2033') 
                                 or (@class = 'ch 2034') 
         ]  
]" >  
  <msub>
    <mrow class="mrow-base"> 
       <xsl:apply-templates select="
             child::mrow[@class = 'mrow-base'] / *
          " />    
       <BREAK SIZE="small"/>
       <xsl:apply-templates select="
             child::mrow[@class = 'mrow-super']
             / child::* [ not(@title = 'speech-extra') ] 
          " />    
    </mrow> 
    <mrow class="mrow-sub"> 
       <xsl:apply-templates 
             select="*[@class = 'mrow-sub'] /* " />
       <xsl:apply-templates select="
             child::mrow[@class = 'mrow-super']
             / child::* [ @class = 'end-script' ] 
          " />    
    </mrow>
  </msub>
</xsl:template> 
>>>



\begin{verbatim}
<span class="msubsup"> 
  <span class="mrow-base"> 
    <span class="mi">x</span> 
  </span> 
  <span class="mrow-sub"> 
    ......
  </span> 
  <span class="mrow-super"> 
    <span class="mid-script" title="speech-extra"> superscript </span> 
    <span class="mi"> 
      <span class="char" title="ch-verbose"> 
        <span class="ch 2032" title="ch-verbose">prime</span> 
      </span> 
    </span> 
  </span> 
</span> 
\end{verbatim}



%%%%%%%%%%%%%
\subsection{Degree}
%%%%%%%%%%%%%

\<remove scrip indicators from degree\><<<
<xsl:template match="mrow[
   (@class = 'mrow-super')
  and
    parent::msup
  and
    (count( `<math content element`> ) = 1)
  and
    `<math content element`> [1][
       normalize-space(.)
       =
       normalize-space( 
          descendant::span[@class = 'ch 2218']
       )
    ]
]" >
  <xsl:copy>
    <xsl:apply-templates select="@*" />
    <xsl:apply-templates 
          select="*[not(@title = 'speech-extra')
                    and
                    not(self::BREAK)
                   ]"
                           mode="degree" />
    <BREAK SIZE="small"/> 
  </xsl:copy>
</xsl:template> 
>>>

\<remove scrip indicators from degree\><<<
<xsl:template match="*" mode="degree" >
  <xsl:copy>
    <xsl:apply-templates select="@*" />
    <xsl:choose>
       <xsl:when test="self::span[ @class='ch 2218' ]" >
         <xsl:text>degree</xsl:text>
       </xsl:when>
       <xsl:otherwise>
         <xsl:apply-templates 
                select="*[not(self::BREAK)]|text()" 
                mode="degree"/>
       </xsl:otherwise>
    </xsl:choose>
   </xsl:copy>
</xsl:template> 
>>>





%%%%%%%%%%%%%
\subsection{Logarithms}
%%%%%%%%%%%%%

\<msub of log\><<<
<xsl:template match="
   msub [
    normalize-space(child::*[1]) = 'log'
   or
    normalize-space(child::*[1]) = 'ln'
]" >
  <xsl:copy>
    <xsl:apply-templates select="@*" />
    <xsl:apply-templates select="child::*[1]" />
    <xsl:apply-templates select="child::*[2]"
                         mode="log" />
  </xsl:copy>
</xsl:template> 
>>>

\<msub of log\><<<
<xsl:template match="*" mode="log" >
  <xsl:copy>
    <xsl:apply-templates select="@*" />
    <xsl:choose>
       <xsl:when test="parent::mrow[ @class = 'mrow-sub' ]" >
         <BREAK SIZE="small"/>
         <xsl:apply-templates select="*" />
         <BREAK SIZE="small"/>
         <span title="speech-extra"> 
             <xsl:text>of</xsl:text>
         </span>          
         <BREAK SIZE="small"/>
       </xsl:when>
       <xsl:otherwise>
          <xsl:apply-templates 
                 select="*[ not(@title = 'speech-extra') ]"
                 mode="log" />
       </xsl:otherwise>
    </xsl:choose>
   </xsl:copy>
</xsl:template> 
>>>

\begin{verbatim}
<msub> 
  <mrow class="mrow-base"> 
    <mo class="MathClass-op">log<BREAK SIZE="small"/></mo> 
  </mrow> 
  <mrow class="mrow-sub"> 
    <span class="begin-script" title="speech-extra"> 
      ...
    </span> 
    <PROS PITCH="-10"> 
      <mn>2</mn> 
    </PROS> 
    <span class="end-script" title="speech-extra"> 
      ...
    </span> 
  </mrow> 
</msub> 
<mi>x</mi> 
\end{verbatim}


%%%%%%%%%%%%%%%%%%
\section{Fractions (Stage 1)}
%%%%%%%%%%%%%%%%%%

%%%%%%%%%%%%%
\subsection{Outline}
%%%%%%%%%%%%%



\<span frac elements\><<<
<script element="mfrac" >
  <set name="mfrac" >
     `<open xslt script`>
     `<frac templates`> 
     `<close xslt script`>
  </set>
  <xslt name="." xml="." xsl="mfrac" />
</script> 
>>>

\<frac templates\><<<
<xsl:template match="mfrac" >
  <xsl:copy>
    <xsl:choose>
       `<word fracs`>
       `<prepend continued fractions`>
       `<tail continued fractions`>
       <xsl:otherwise>
         <xsl:apply-templates select="*|@*|text()|comment()" />
       </xsl:otherwise>
    </xsl:choose>
  </xsl:copy>
</xsl:template> 
>>>


%%%%%%%%%%%%%
\subsection{Word Fractions}
%%%%%%%%%%%%%

Example: `\verb+1 \ove 2+' int `one half'

\<word fracs\><<<
<xsl:when test=" 
   (string-length(
      normalize-space(child::mrow[ @class = 'mrow-numerator' ][1])
     ) = 1)
   and
   (string-length(
     normalize-space(child::mrow[ @class = 'mrow-enumerator' ][1])
     ) = 1)
">
   `<a := numerator; b := enumerator`>
   <xsl:choose>
      <xsl:when test=" 
           (translate($a,'123456789','') != '')
           or
           (translate($b,'123456789','') != '')
       " >
         <xsl:apply-templates select="*|@*|text()|comment()" />
      </xsl:when>
      <xsl:when test=" $a &lt; $b ">
         <xsl:attribute  name="class">
             <xsl:text>word-frac</xsl:text>
         </xsl:attribute>
         `<word numerator`>
         `<word enumerator`>
      </xsl:when>
      <xsl:otherwise>
         <xsl:apply-templates select="*|@*|text()|comment()" />
      </xsl:otherwise>
   </xsl:choose>    
</xsl:when>
>>>


\<a := numerator; b := enumerator\><<<
<xsl:variable name="a">
   <xsl:value-of select="
      normalize-space(child::mrow[ @class = 'mrow-numerator' ][1])
   " />
</xsl:variable>
<xsl:variable name="b">
   <xsl:value-of select="
      normalize-space(child::mrow[ @class = 'mrow-enumerator' ][1])
   " />
</xsl:variable>
>>>


\<word numerator\><<<
<xsl:choose>
   <xsl:when test=" $a = 1 "><xsl:text> one </xsl:text></xsl:when>
   <xsl:when test=" $a = 2 "><xsl:text> two </xsl:text></xsl:when>
   <xsl:when test=" $a = 3 "><xsl:text> three </xsl:text></xsl:when>
   <xsl:when test=" $a = 4 "><xsl:text> four </xsl:text></xsl:when>
   <xsl:when test=" $a = 5 "><xsl:text> five </xsl:text></xsl:when>
   <xsl:when test=" $a = 6 "><xsl:text> six </xsl:text></xsl:when>
   <xsl:when test=" $a = 7 "><xsl:text> seven </xsl:text></xsl:when>
   <xsl:when test=" $a = 8 "><xsl:text> eight </xsl:text></xsl:when>
   <xsl:when test=" $a = 9 "><xsl:text> nine </xsl:text></xsl:when>
</xsl:choose>
>>>


\<word enumerator\><<<
<xsl:choose>
   <xsl:when test=" $b = 2 "><xsl:text> half</xsl:text></xsl:when>
   <xsl:when test=" $b = 3 "><xsl:text> third</xsl:text></xsl:when>
   <xsl:when test=" $b = 4 "><xsl:text> fourth</xsl:text></xsl:when>
   <xsl:when test=" $b = 5 "><xsl:text> fifth</xsl:text></xsl:when>
   <xsl:when test=" $b = 6 "><xsl:text> sixth</xsl:text></xsl:when>
   <xsl:when test=" $b = 7 "><xsl:text> seventh</xsl:text></xsl:when>
   <xsl:when test=" $b = 8 "><xsl:text> eighth</xsl:text></xsl:when>
   <xsl:when test=" $b = 9 "><xsl:text> nineth</xsl:text></xsl:when>
</xsl:choose>
<xsl:if test=" $a &gt; 1 "><xsl:text>s</xsl:text></xsl:if>
<xsl:text> </xsl:text>
>>>



%%%%%%%%%%%%%
\subsection{Mixed Fractions}
%%%%%%%%%%%%%

Examples: `\verb+2{3\over 4}+' 
into `2 and three forth',
and `\verb+10{25\over 39}+' into `10 and 25 over 39'.

\<mixed fractions\><<<
<xsl:template match="mfrac[
   ( 
      (translate(
         concat(
           mrow[ (@class = 'mrow-numerator')],
           mrow[ (@class = 'mrow-enumerator')] 
         )  ,'0123456789','') = '')
      and
         not(descendant::*/descendant::*
                          /descendant::mn-group-s)
      or 
         (@class = 'word-frac')
   )
   and 
     preceding-sibling::*[1]
       / self::mn-group-s[ child::*[
            (position() = last())
            and
            self::mn
         ] ]
 ]" >
  <xsl:text> and </xsl:text>
  <xsl:copy>
    <xsl:apply-templates select="*|@*|text()|comment()" />
  </xsl:copy>
</xsl:template> 
>>>




%%%%%%%%%%%%%
\subsection{Continued Fractions Conditions (Rule 69)}
%%%%%%%%%%%%%



\begin{verbatim}
          1 
1 + ---------------------------------------
              1
    2 + ---------------------------------------
                  1 
        2 + ---------------------------------------
              2 + ...
\end{verbatim}

 
The fractions are processed recusrively from inside out.  The `tail'
is the one to discover the inner most part of a continued fraction
(i.e., the bottom rows).  Higher levels add to the nucleous built
already under them.

In Stage 1 the members of the continued fractions are marked as
such. In the second stage they being are procesed.


The minimum conditions here are three un-interrupted levels of
fractions.

\begin{verbatim}
<mfrac>
   <mrow class="mrow-enumerator"> 
      <mn>1</mn> 
      <mo class="MathClass-bin">...</mo> 
      <mfrac> 
        <span class="begin-end" ...>begin fraction</span> 
        <mrow class="mrow-numerator"> ... </mrow>
        <span class="begin-end" ...>over</span> 
        <mrow class="mrow-enumerator">...</mrow> 
        <span class="begin-end" ...>end fraction</span> 
      </mfrac>
\end{verbatim}

\<a,b := top 2 pre op values\><<<
<xsl:variable name="a">
   <xsl:apply-templates select="
         child::mrow[ @class = 'mrow-enumerator' ] /
         child::mfrac /
         preceding-sibling::*[2] 
   "  mode="enum-op" />  
</xsl:variable>
<xsl:variable name="b">
   <xsl:apply-templates select="
         child::mrow[ @class = 'mrow-enumerator' ] /
         child::mfrac /
         child::mrow[ @class = 'mrow-enumerator' ] /
         child::mfrac /
         preceding-sibling::*[2] 
   "  mode="enum-op" />  
</xsl:variable>
<xsl:variable name="c">
   <xsl:value-of select="
    normalize-space(
      child::mrow[ @class = 'mrow-enumerator' ] /
      child::mfrac /
      child::mrow[ @class = 'mrow-enumerator' ] /
      child::mfrac /
      child::mrow[ @class = 'mrow-enumerator' ] )
   "  />  
</xsl:variable>
>>>

\<frac templates\><<<
<xsl:template match="*" mode="enum-op">
   <xsl:if test="preceding-sibling::*" >
      <xsl:apply-templates select=" preceding-sibling::*[1] " />
   </xsl:if>
   <xsl:value-of select="." />
</xsl:template>
>>>





%%%%%%%%%%%%%
\subsection{Nucleous of Continued Fractions}
%%%%%%%%%%%%%

We are looking for the three most internal (bottom) fractions that
belong to a continued fraction.


\<tail continued fractions\><<<
<xsl:when test="          `%check numeral numerator`%
   (translate(
      normalize-space(
      child::mrow[ @class = 'mrow-numerator' ]),
                              '0123456789','')= '')
   and `<check equality of numerators`>
   and `<check ops before two top fracs`>
">
  `<a,b := top 2 pre op values`>
  <xsl:choose>
     <xsl:when test="
         ( translate($a,'0123456789 ','') = '')
         and (normalize-space($a)=normalize-space($b)) 
         and starts-with( $c, normalize-space( $a )) 
         and starts-with(
               normalize-space(
                 substring-after( $c, normalize-space( $a )) )
               , 
               normalize-space(
                    child::mrow[ @class = 'mrow-enumerator' ] /
                    child::mfrac /
                    preceding-sibling::*[1]                  )
             )
     " >
         <xsl:attribute name="class">
            <xsl:value-of select=" 'continued-mfrac' " />
         </xsl:attribute>         
         <xsl:apply-templates select="*|text()|comment()" />
     </xsl:when>
     <xsl:otherwise>
         <xsl:apply-templates select="*|@*|text()|comment()" />
     </xsl:otherwise>
  </xsl:choose>
</xsl:when>
>>>





\<check equality of numerators\><<<
(  normalize-space(
   child::mrow[ @class = 'mrow-numerator' ])
  and
   normalize-space(
   child::mrow[ @class = 'mrow-enumerator' ] /
   child::mfrac /
   child::mrow[ @class = 'mrow-numerator' ])
)
and
(  normalize-space(
   child::mrow[ @class = 'mrow-enumerator' ] /
   child::mfrac /
   child::mrow[ @class = 'mrow-numerator' ])
  and
   normalize-space(
   child::mrow[ @class = 'mrow-enumerator' ] /
   child::mfrac /
   child::mrow[ @class = 'mrow-enumerator' ] /
   child::mfrac /
   child::mrow[ @class = 'mrow-numerator' ])
)
>>>



\<check ops before two top fracs\><<<
(  child::mrow[ @class = 'mrow-enumerator' ] /
   child::mfrac /
   preceding-sibling::*[1][@class = 'MathClass-bin']
)
and  
(  normalize-space(
   child::mrow[ @class = 'mrow-enumerator' ] /
   child::mfrac /
   preceding-sibling::*[1] )
   =
   normalize-space(
   child::mrow[ @class = 'mrow-enumerator' ] /
   child::mfrac /
   child::mrow[ @class = 'mrow-enumerator' ] /
   child::mfrac /
   preceding-sibling::*[1] )
)
>>>




%%%%%%%%%%%%%
\subsection{Prepend Existing Continued Fractions}
%%%%%%%%%%%%%


\<prepend continued fractions\><<<
<xsl:when test="
      self::mfrac
    and
      child::mrow[ @class = 'mrow-enumerator' ]
        / child::mfrac[ @class = 'continued-mfrac' ]
    and
      ( normalize-space(
          child::mrow[ @class = 'mrow-numerator' ]
        )
        =
        normalize-space(
          child::mrow[ @class = 'mrow-enumerator' ]
           / child::mfrac[ @class = 'continued-mfrac' ]
           / child::mrow[ @class = 'mrow-numerator' ]  
        )
      ) 
    and `<check equality of op with cont frac`>
" >
  `<a,b := cont top 2 pre op values`>
  <xsl:choose>
     <xsl:when test="
         normalize-space($a)=normalize-space($b)
     " >
         <xsl:attribute name="class">
            <xsl:value-of select=" 'continued-mfrac' " />
         </xsl:attribute>         
         <xsl:apply-templates select="*|text()|comment()" />
     </xsl:when>
     <xsl:otherwise>
         <xsl:apply-templates select="*|@*|text()|comment()" />
     </xsl:otherwise>
  </xsl:choose>
</xsl:when> 
>>>



\<check equality of op with cont frac\><<<
(
   normalize-space(
   child::span[ @class = 'mrow-enumerator' ] /
   child::mfrac /
   preceding-sibling::*[1] )
   =
   normalize-space(
   child::span[ @class = 'mrow-enumerator' ] /
   child::mfrac /
   child::span[ @class = 'mrow-enumerator' ] /
   child::span[ @class = 'continued-mfrac' ] /
   preceding-sibling::*[1] )
)
>>>


\<a,b := cont top 2 pre op values\><<<
<xsl:variable name="a">
   <xsl:apply-templates select="
         child::span[ @class = 'mrow-enumerator' ] /
         child::span[ @class = 'mcontinued-mfrac' ] /
         preceding-sibling::*[2] 
   "  mode="enum-op" />  
</xsl:variable>
<xsl:variable name="b">
   <xsl:apply-templates select="
         child::span[ @class = 'mrow-enumerator' ] /
         child::span[ @class = 'mcontinued-mfrac' ] /
         child::span[ @class = 'mrow-enumerator' ] /
         child::span[ 
           self::mfrac or (@class = 'mcontinued-mfrac')
         ] /
         preceding-sibling::*[2] 
   "  mode="enum-op" />  
</xsl:variable>
>>>




%%%%%%%%%%%%%%%%%%
\section{Fractions (Stage 2)}
%%%%%%%%%%%%%%%%%%

%%%%%%%%%%%%%
\subsection{Outline}
%%%%%%%%%%%%%


\<set levels for hyper complex fracs\><<<
<dom name="." xml="." method="fracLevel" class="tex4ht.HtJsml" />
`<remove xml declaration`>
>>>



\<static void fracLevel(dom)\><<<
public static void fracLevel(Node d) {
   dom = (Document) d;
   setFracLevel(dom.getFirstChild(), 0);
}
private static int setFracLevel(Node node, int cont) {
  int level = 0;
  if (node.hasChildNodes()) {
        String ndName = node.getNodeName();
    int prevCont = cont;
    String clValue = null;
    if (ndName.equals("mfrac")) {        
      `<clValue := class of mfrac`>
      `<cont := distance from mfrac(continued-frac)`>
    }
    `<level += inherited from children`>
    `<return 0 if barier`>
    if (ndName.equals("mfrac")) {
      `<return 0 if mfrac barier`>
      if( cont > 0 ){
          if ( prevCont == 0 ){
             `<set start continued fracs mark`>
          }
          `<remove end of non-tail continued frac`> 
          level = 0;
      } else if ( prevCont > 0 ){
          `<set end continued fracs mark`>
          level = 0;
      } else
        if( level > 0 ){
          `<set extra levels for frac`>
        }
        level++;
  } }
  return level;
}
>>>



%%%%%%%%%%%%%%%%%%
\subsection{Level Indicators for Nesting (Hyper Complex Fractions)}
%%%%%%%%%%%%%%%%%%

\begin{verbatim}
                  ${a\over b}\over c$ 

begin begin fraction
                     begin fraction   a 
                     over             b
                     end fraction
over over          c 
end end fraction
\end{verbatim}

\<set extra levels for frac\><<<
for (int i = 0; i < children.getLength(); i++) {
   Node child = children.item(i);
   if (child.getNodeType() == Node.ELEMENT_NODE) {
      Node cls = child.getAttributes()
                      .getNamedItem("class");
      if (cls != null) {
         String clsName = cls.getNodeValue();
         if (clsName.equals("begin-end")) {
            insertLevelPrefix(child, level);
}  }  }  }
>>>

\<HtJsml utility members\><<<
private static void insertLevelPrefix(Node node, int level){
   if( level == 0 ){ return; }
   if (node.getNodeType() == Node.ELEMENT_NODE) {
     if( node.getNodeName().equals( "level" ) ){
        `<insert level prefixes`>
     } else {
        NodeList children = node.getChildNodes();
        for (int i = 0; i < children.getLength(); i++) {
           Node child = children.item(i);
           insertLevelPrefix(child, level);
}  }  } }
>>>


   

\<insert level prefixes\><<<
Node attr = node.getAttributes().getNamedItem("prefix");
if( attr != null ){
  String prefix = attr.getNodeValue();
  String  s = "";
  for(int j=0; j<level; j++){ 
     s += prefix + " ";
  }
  ((org.w3c.dom.Element) node).setAttribute( "depth", ""+level);
  Node child = node.getFirstChild();
  if( child != null ){
     node.insertBefore( dom.createTextNode(s), child );
} }
>>>



\<level += inherited from children\><<<
NodeList children = node.getChildNodes();
int max = 0;
for (int i = 0; i < children.getLength(); i++) {
   Node child = children.item(i);
   if (child.getNodeType() == Node.ELEMENT_NODE) {
      int d = setFracLevel(child, 
                           `<continued level counter`>);
      if (d > max) { max = d; }
}  }
level += max;
>>>


%%%%%%%%%%%%%
\subsection{Level Terminators: Scripts, Word Fracs}
%%%%%%%%%%%%%


\<return 0 if barier\><<<
if(    ndName.equals("msub")
    || ndName.equals("msup") 
    || ndName.equals("msubsup") 
) {
   return 0;
}
>>>


\<return 0 if mfrac barier\><<<
if( (clValue != null) && clValue.equals("word-frac")
) {
   return 0;
}
>>>










%%%%%%%%%%%%%
\subsection{Bookeeping for Continued Fractions}
%%%%%%%%%%%%%

\<continued level counter\><<<
cont
>>>


In the first phase, the mfrac elements in a group carry 
attributes as in

\begin{verbatim}
       <mfrac class="continued-mfrac"> 
\end{verbatim}


with the exception of the bottom (internal) most leveles.
The `cont' parameter is for determining whether the parent and the
grandparent are frac elements marked as continued.



\<clValue := class of mfrac\><<<
if (node.hasAttributes()) {
   Node cl = node.getAttributes().getNamedItem("class");
   if (cl != null) { clValue = cl.getNodeValue(); }
}
>>>

\<cont := distance from mfrac(continued-frac)\><<<
if( (clValue != null) 
    && clValue.equals("continued-mfrac") ) {
  cont = 2; 
} else { cont--; }
>>>



%%%%%%%%%%%%%
\subsection{Modifications for Continued Fractions}
%%%%%%%%%%%%%

\<remove end of non-tail continued frac\><<<
Node child = node.getLastChild();
if (child.getNodeType() == Node.ELEMENT_NODE) {
   Node cls = child.getAttributes() .getNamedItem("class");
   if (cls != null) {
      String clsName = cls.getNodeValue();
      if ( clsName.equals("begin-end")) {
         node.removeChild( child  );
}  }  }
>>>

\<set end continued fracs mark\><<<
Node child = node.getLastChild();
if (child.getNodeType() == Node.ELEMENT_NODE) {
   Node cls = child.getAttributes() .getNamedItem("class");
   if (cls != null) {
      String clsName = cls.getNodeValue();
      if ( clsName.equals("begin-end")) {
         setContinuedNote(child);
}  }  }
>>>

\<set start continued fracs mark\><<<
Node child = node.getFirstChild();
if (child.getNodeType() == Node.ELEMENT_NODE) {
   Node cls = child.getAttributes() .getNamedItem("class");
   if (cls != null) {
      String clsName = cls.getNodeValue();
      if ( clsName.equals("begin-end")) {
         setContinuedNote(child);
}  }  }
>>>


\<HtJsml utility members\><<<
private static void setContinuedNote(Node node){
   if (node.getNodeType() == Node.ELEMENT_NODE) {
     if( node.getNodeName().equals( "level" ) ){
        `<fix end continued note`>
     } else {
        NodeList children = node.getChildNodes();
        for (int i = 0; i < children.getLength(); i++) {
           Node child = children.item(i);
           setContinuedNote(child);
}  }  } }
>>>

\<fix end continued note\><<<
Node attr = node.getAttributes().getNamedItem("continued");
if (attr != null) {
  node = node.getFirstChild();
  if( node != null ){
     ((org.w3c.dom.Text) node).setData( attr.getNodeValue() );
} }
>>>




%%%%%%%%%%%%%
\section{Set Levels on Roots}
%%%%%%%%%%%%%

\<set levels for roots\><<<
<dom name="." xml="." method="rootLevel" class="tex4ht.HtJsml" />
`<remove xml declaration`>
>>>

\<static void rootLevel(dom)\><<<
public static void rootLevel(Node d) {
   dom = (Document) d;
   setRootLevel(d.getFirstChild());
}
private static int setRootLevel( Node node ){
  int level = 0;
//  String clName = null;
  if (node.hasChildNodes()) {
    `<count root levels inherited from children`>
    String ndName =  node.getNodeName();
    `<block for roots`>
    if( ndName.equals("msqrt") || ndName.equals("mroot") ){
        `<set extra levels for roots`>
        level++;
  } } 
  return level;
}
>>>


\<count root levels inherited from children\><<<
NodeList children = node.getChildNodes();
int max = 0;
for (int i = 0; i < children.getLength(); i++) {
  Node child = children.item(i);
  if (child.getNodeType() == Node.ELEMENT_NODE) {
    int d = setRootLevel(child);
    if( d > max ){ max = d; }
} }
level += max;
>>>


\<block for roots\><<<
if( ndName.equals("msub") || ndName.equals("msup") ||
    ndName.equals("msubsup") 
) {
   return 0;
}
>>>




\<set extra levels for roots\><<<
for (int i = 0; i < children.getLength(); i++) {
  Node child = children.item(i);
  if (child.getNodeType() == Node.ELEMENT_NODE) {
    Node cls = child.getAttributes()
                    .getNamedItem("class");
    if (cls != null) {
      String clsName = cls.getNodeValue();
      if( clsName.equals("begin-root") 
          || clsName.equals("mid-root") 
          || clsName.equals("end-root") 
      ){
         insertLevelPrefix(child, level);
} } } }
>>>




%%%%%%%%%%%%%
\section{Modifiers}
%%%%%%%%%%%%%



%%%%%%%%%%%%%%%%%%
\subsection{Over and Under Limit Scripts}
%%%%%%%%%%%%%%%%%%


\<over and under scripts\><<<
<script element="msub::limits-msub-msup" >
  <set name="smash" >
     `<open xslt script`>
     `<smash over and under scripts`>
     `<close xslt script`>
  </set>
  <xslt name="." xml="." xsl="smash" />
</script> 
>>>

\<over and under scripts\><<<
<script element="msup::limits-msub-msup" >
  <set name="smash" >
     `<open xslt script`>
     `<smash over and under scripts`>
     `<close xslt script`>
  </set>
  <xslt name="." xml="." xsl="smash" />
</script> 
>>>

\<over and under scripts\><<<
<script element="msubsup::limits-msub-msup" >
  <set name="smash" >
     `<open xslt script`>
     `<smash over and under scripts`>
     `<close xslt script`>
  </set>
  <xslt name="." xml="." xsl="smash" />
</script> 
>>>







\<smash over and under scripts\><<<
<xsl:template match="*[
       (@class = 'limits-msub-msup')
     and
       child::*[ (position() = 1) 
                 and (@class = 'limits-mrow-base')
                 and child::*[ (position() = 1) 
                         and (@class = 'limits-msub-msup')
       ]             ]
]" >
  <xsl:copy>
     <xsl:apply-templates select="@*" />
     <xsl:apply-templates select="
                   *[1]
                   / *[1] 
                   / *[ not(@class='limits-mrow-super') ]  " />
     <xsl:apply-templates select=" *[
                   preceding-sibling::* 
                 and
                   not(@class='limits-mrow-super') ]" />
     <xsl:apply-templates select="
                   *[1]
                   / *[1]
                   / *[@class='limits-mrow-super' ]  " />
     <xsl:apply-templates select=" *[
                   preceding-sibling::* 
                 and
                   (@class='limits-mrow-super') ]" />
  </xsl:copy>
</xsl:template> 
>>>


\<compress limit script\><<<
<xsl:template match="span[ @class = 'end-limits-script' ]" >
  <xsl:if test=" parent::*[ not(following-sibling::*) ] ">
    <xsl:copy>
       <xsl:apply-templates select=" @* " />
       <xsl:choose>
          <xsl:when test="  
                parent::*[ preceding-sibling::mrow[
                                  @class != 'limits-mrow-base'
                         ]  ]
            " >
            <BREAK SIZE="small"/>
            <xsl:text> end scripts </xsl:text>
            <BREAK SIZE="small"/>
          </xsl:when>
          <xsl:otherwise>
            <BREAK SIZE="small"/>
            <xsl:text> end script </xsl:text>
            <BREAK SIZE="small"/>
          </xsl:otherwise>
       </xsl:choose>
    </xsl:copy>
  </xsl:if>
</xsl:template> 
>>>


\<compress limit script\><<<
<xsl:template match="span[ @class = 'begin-limits-script' ]" >
    <xsl:copy>
       <xsl:apply-templates select=" @* " />
       <xsl:choose>
          <xsl:when test="  
                parent::*[ @class = 'limits-mrow-super'  ]
          " >
              <xsl:apply-templates 
                  select=" parent::* 
                           / preceding-sibling::*[1] " 
                  mode = "extra-over" />
          </xsl:when>
          <xsl:when test="  
                parent::*[ @class = 'limits-mrow-sub'  ]
          " >
              <xsl:apply-templates 
                  select=" parent::* 
                           / preceding-sibling::*[1] " 
                  mode = "extra-under" />
          </xsl:when>
       </xsl:choose>
       <xsl:apply-templates select="*|text()|comment()" />
    </xsl:copy>
</xsl:template> 
>>>



\<compress limit script\><<<
<xsl:template match="*" mode="extra-over" >
   <xsl:if   test = " self::mrow[ @class = 'limits-mrow-super' ] " >
     <xsl:text> over </xsl:text>
     <xsl:apply-templates select=" preceding-sibling::*[1] " 
                    mode = "extra-over" />
   </xsl:if>
</xsl:template> 
>>>


\<compress limit script\><<<
<xsl:template match="*" mode="extra-under" >
   <xsl:if   test = " self::mrow[ @class = 'limits-mrow-sub' ] " >
     <xsl:text> under </xsl:text>
     <xsl:apply-templates select=" preceding-sibling::*[1] " 
                    mode = "extra-under" />
   </xsl:if>
</xsl:template> 
>>>




%%%%%%%%%%%%%
\subsection{Short Cuts for Math Underline}
%%%%%%%%%%%%%



\<short cut modifiers\><<<
<script element="munder::munder-underline" >
  <set name="munder" >
     `<open xslt script`>
     `<get content template`>
     `<under modifier templates`> 
     `<close xslt script`>
  </set>
  <xslt name="." xml="." xsl="munder" />
</script> 
>>>

\<under modifier templates\><<<
<xsl:template match="munder[
    (@class = 'munder-underline')
   and
    child::mrow[ 
       (@class = 'mo-0332')
       and
       descendant::mi
   ]
]" >
    <xsl:variable name="content">
       <xsl:apply-templates select="*" mode="content" />
    </xsl:variable>
    <xsl:choose>
       <xsl:when test=" 
          string-length( normalize-space( $content )) = 1
       " >       
           <xsl:copy>
              <xsl:apply-templates select="@*" />
              <xsl:apply-templates
                 select=" *[ @class != 'begin-end' ] " />
              <span class="begin-end" title="speech-extra" > 
                 <BREAK SIZE="small"/>
                 <xsl:text> under bar </xsl:text>
                 <BREAK SIZE="small"/>
              </span>
           </xsl:copy>
       </xsl:when>
       <xsl:otherwise>
          `<set munder`>
       </xsl:otherwise>
    </xsl:choose>
</xsl:template> 
>>>




The following code makes to push the munder's under the mover's.

\<set munder\><<<
<xsl:choose>
   <xsl:when test="child::mrow / child::mover">
         <xsl:apply-templates select="child::mrow / child::mover"
                                mode="under-mover" />
   </xsl:when>
   <xsl:otherwise>
     <xsl:copy>
        <xsl:apply-templates select=" *|@*|text() " />
     </xsl:copy>
   </xsl:otherwise>  
</xsl:choose>
>>>

\<under modifier templates\><<<
<xsl:template match="*" mode="under-mover" >
   <xsl:choose>
      <xsl:when test="self::mover">
         <xsl:copy>
           <xsl:apply-templates select="@*" />
           <xsl:apply-templates select="*" mode="under-mover" />
         </xsl:copy>
      </xsl:when>
      <xsl:when test="self::mrow[ parent::mover ]">
         <xsl:copy>
           <xsl:apply-templates select="@*" />
           `<cont under-mover`>
         </xsl:copy>
      </xsl:when>
      <xsl:otherwise>
         <xsl:apply-templates select="." />
      </xsl:otherwise>
   </xsl:choose>
</xsl:template>
>>>

\<cont under-mover\><<<
<xsl:choose>
   <xsl:when test="child::mover">
      <xsl:apply-templates select="*" mode="under-mover" />
   </xsl:when>
   <xsl:otherwise>
      <munder class="munder-underline"> 
         `<munder prefix`>
         <mrow class="mo-0032">
            <xsl:apply-templates select="*"  />
         </mrow>
         `<munder postfix`>
      </munder>
   </xsl:otherwise>
</xsl:choose>
>>>



\<munder prefix\><<<
<xsl:apply-templates select="
  ancestor::munder[1] / child::*[following-sibling::mrow]
"/>
>>>



\<munder postfix\><<<
<xsl:apply-templates select="
  ancestor::munder[1] / child::*[preceding-sibling::mrow]
"/>
>>>



\begin{verbatim}
<munder class="munder-underline">
  <span class="begin-end" title="speech-extra" > modified under </span>
  <mrow class="mo-0332">
    <mi>x</mi>
  </mrow>
  <span class="begin-end" title="speech-extra" > with bar </span>
</munder>
\end{verbatim}



%%%%%%%%%%%%%
\subsection{Short Cuts for Math Overline}
%%%%%%%%%%%%%


\<short cut modifiers\><<<
<script element="mover::mover-overline" >
  <set name="mover" >
     `<open xslt script`>
     `<get content template`>
     `<over modifier templates`> 
     `<close xslt script`>
  </set>
  <xslt name="." xml="." xsl="mover" />
</script> 
>>>




\<over modifier templates\><<<
<xsl:template match="mover[
    (@class = 'mover-overline')
   and
    child::mrow[ 
       (@class = 'mo-00AF')
       and
       descendant::mi
   ]
]" >
    <xsl:variable name="content">
       <xsl:apply-templates select="*" mode="content" />
    </xsl:variable>
    <xsl:choose>
       <xsl:when test=" 
          string-length( normalize-space( $content )) = 1
       " >       
          `<set mover short cut`>
       </xsl:when>
       <xsl:otherwise>
          <xsl:copy>
             <xsl:apply-templates select="@*" />
             <xsl:apply-templates
                     select="*|text()|comment()" />
          </xsl:copy>
       </xsl:otherwise>
    </xsl:choose>
</xsl:template> 
>>>


The following code makes to push the mover's under the munder's.

\<set mover short cut\><<<
<BREAK SIZE="small"/>
<xsl:choose>
   <xsl:when test="child::mrow / child::munder">
         <xsl:apply-templates select="child::mrow / child::munder"
                       mode="under-munder" />
   </xsl:when>
   <xsl:otherwise>
     <xsl:copy>
       <xsl:apply-templates select="@*" />
       <xsl:apply-templates
           select=" *[ @class != 'begin-end' ] " />
         `<mover bar short cut`>
     </xsl:copy>
   </xsl:otherwise>  
</xsl:choose>
>>>

\<over modifier templates\><<<
<xsl:template match="*" mode="under-munder" >
   <xsl:choose>
      <xsl:when test="self::munder">
         <xsl:copy>
           <xsl:apply-templates select="@*" />
           <xsl:apply-templates select="*" mode="under-munder" />
         </xsl:copy>
      </xsl:when>
      <xsl:when test="self::mrow[ parent::munder ]">
         <xsl:copy>
           <xsl:apply-templates select="@*" />
           `<cont under-munder`>
         </xsl:copy>
      </xsl:when>
      <xsl:otherwise>
         <xsl:apply-templates select="." />
      </xsl:otherwise>
   </xsl:choose>
</xsl:template>
>>>

\<cont under-munder\><<<
<xsl:choose>
   <xsl:when test="child::munder">
      <xsl:apply-templates select="*" mode="under-munder" />
   </xsl:when>
   <xsl:otherwise>
      <mover class="mover-overline"> 
         <mrow class="mo-00AF">
            <xsl:apply-templates select="*"  />
         </mrow>
         `<mover bar short cut`>
      </mover>
   </xsl:otherwise>
</xsl:choose>
>>>





\<mover bar short cut\><<<
<span class="begin-end" title="speech-extra" > 
   <BREAK SIZE="small"/>
   <xsl:text> over bar </xsl:text>
   <BREAK SIZE="small"/>
</span>
>>>




\begin{verbatim}
<mover class="mover-overline">
  <span class="begin-end" title="speech-extra"> modified above </span>
  <mrow class="mo-00AF">
    <span class="mathvariant-bold">
      <span title="speech-extra" class="begin-end"> bold </span>
      <mi class="mi">
          <span title="speech-extra" 
                class="capital-description"> capital </span>
          Z
      </mi>
    </span>
  </mrow>
  <span class="begin-end" title="speech-extra"> with bar </span>
</mover>
\end{verbatim}







%%%%%%%%%%%%%
\section{New Theorems}
%%%%%%%%%%%%%

\<boundaries on theorems\><<<
<script element="div::newtheorem" >
  <set name="newtheorem" >
     `<open xslt script`>
     `<annotate bounderies of theorems`> 
     `<tags for empty templates`>
     `<close xslt script`>
  </set>
  <xslt name="." xml="." xsl="newtheorem" />
</script> 
>>>


\<annotate bounderies of theorems\><<<
<xsl:template match=" 
    div[   (@class='newtheorem') 
         and
           descendant::*[ self::p ][1] 
           / descendant::*[ self::span ][1]
                          [ @class = 'theorem-head' ]
    ]
" >
   <xsl:copy>
     <xsl:apply-templates select="*|@*|text()|comment()" />
     <div class="begin-end" title="speech-extra"> 
        <BREAK SIZE="small"/>
        <xsl:value-of select=" 
           concat( ' end ', 
              substring-before(`<new theorem header`>, ' ')
           ) " />
     </div>
   </xsl:copy>
</xsl:template> 
>>>


\<new theorem header\><<<
concat(
  normalize-space(
    string( 
       descendant::*[ self::p ][1] 
       / descendant::*[ self::span ][1]
                      [ @class = 'theorem-head' ]
  ) )
  , ' '
)
>>>


%%%%%%%%%%%%%%%%%%
\section{Shared}
%%%%%%%%%%%%%%%%%%



\<open xslt script\><<<
<![CDATA[ 
   <xsl:stylesheet version="1.0"
      xmlns:xsl="http://www.w3.org/1999/XSL/Transform"
   >
      <xsl:output omit-xml-declaration = "yes" />
>>>

\<close xslt script\><<<
      <xsl:template match="*|@*|text()|comment()" >
        <xsl:copy>
          <xsl:apply-templates select="*|@*|text()|comment()" />
        </xsl:copy>
      </xsl:template>
   </xsl:stylesheet> 
]]>
>>>


\<get content template\><<<
<xsl:template match="*" mode="content" >
  <xsl:choose>
     <xsl:when test=" @class = 'char' " >   
       <xsl:if test="not( child::*[
            (@class = 'ch 2032') or (@class = 'ch 2033') 
                                 or (@class = 'ch 2034') 
       ] )">
         <xsl:text>x</xsl:text>
       </xsl:if>
     </xsl:when>
     <xsl:when test=" not( 
            (@title = 'speech-extra') or (@class = 'accent-char')
         ) " >
       <xsl:apply-templates select="*|text()" mode="content" />
     </xsl:when>
  </xsl:choose>
</xsl:template> 
>>>





\marginpar{Can dom be prevented from creating an xml declaration in
  the output? The xslt part is there just to remove the undesirable
  declaration.}


\<remove xml declaration\><<<
<set name="rmXmlDecl" >
  `<open xslt script`>
  `<close xslt script`>
</set>
<xslt name="." xml="." xsl="rmXmlDecl" />
>>>









\AtEndDocument{\Needs{%
    "cd /home/4/gurari/xtpipes.dir
     ;  
     jar cf tex4ht.jar *
     ;
     mv tex4ht.jar /home/4/gurari/tex4ht.dir/texmf/tex4ht/bin/. 
     ;
     cp /home/4/gurari/xtpipes.dir/xtpipes/lib/* 
        /home/4/gurari/tex4ht.dir/texmf/tex4ht/xtpipes/.
"}}



%%%%%%%%%%%%%%%%%%%%%%%%%%%%%%%%%%%%%%%%%%%%%%%%
\end{document}
%%%%%%%%%%%%%%%%%%%%%%%%%%%%%%%%%%%%%%%%%%%%%%%%



%%%%%%%%%%%%%%%%%%
\section{Odd Ends}
%%%%%%%%%%%%%%%%%%

%%%%%%%%%%%%%
\subsection{Font Decorations}
%%%%%%%%%%%%%

\<bold math\><<<
<script element="span::mathvariant-bold" >
  <set name="bold" >
     `<open xslt script`>
     `<get content template`>
     `<math bold templates`> 
     `<close xslt script`>
  </set>
  <xslt name="." xml="." xsl="bold" />
</script> 
>>>



\<math bold templates\><<<
<xsl:template match="span[ @class = 'mathvariant-bold' ]" >
  <xsl:copy>
    <xsl:apply-templates select="@*" />
    <xsl:variable name="content">
       <xsl:apply-templates select="*" mode="content" />
    </xsl:variable>
    <xsl:choose>
       <xsl:when test=" 
          string-length( normalize-space( $content )) = 1
       " >
          <span class="begin-end" title="speech-extra" > 
             <xsl:text> bold </xsl:text>
          </span>
          <xsl:apply-templates select="*|text()|comment()" />
       </xsl:when>
       <xsl:otherwise>
          <span class="begin-end" title="speech-extra" > 
             <xsl:text> begin bold </xsl:text>
          </span>
          <xsl:apply-templates select="*|text()|comment()" />
          <span class="begin-end" title="speech-extra" > 
             <xsl:text> end bold </xsl:text>
          </span>
       </xsl:otherwise>
    </xsl:choose>
  </xsl:copy>
</xsl:template> 
>>>

%%%%%%%%%%%%%%%%%%
\section{Empty Elements}
%%%%%%%%%%%%%%%%%%



%%%%%%%%%%%%%%%%%%
\section{Prose}
%%%%%%%%%%%%%%%%%%
