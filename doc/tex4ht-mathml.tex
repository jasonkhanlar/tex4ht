% $Id: tex4ht-mathml.tex 117 2013-10-09 17:45:44Z karl $
% Copyright 2009-2013 CV Radhakrishnan.
% Released under LPPLv1.3+.
% 
% TeX4ht and MathML 3.

\ifx\texhtstandalonedoc1
\documentclass[a4paper]{article}
\usepackage{xspace,blog}
\begin{document}
\section{MathML 3 and implications on TeX4ht}
\else
\chapter{MathML 3 and implications for \TeX4ht}
\fi

\leavevmode \hlink{http://www.w3.org/TR/MathML3/}{\mathml 3} is about to
be formalized and going to be released as the new standard for encoding
mathematics in web. It differs from the previous standard version 2. The
main changes that have a bearing on the functionality of
\hlink{http://www.tug.org/tex4ht/}{\texht} are discussed here.

\section{Linking}

\mathml 1 and 2 didn't have any built in provisions to specify links for
various elements. The recommendation was to use
\hlink{http://www.w3.org/TR/xlink/}{\xlink} since it was part of the
original grand plan for \hlink{http://www.w3.org/standards/xml/}{\xml}
in its formative years. However, \xlink didn't gain much popularity nor
was it supported widely. Therefore, \mathml 3 now specifies that any
\mathml element can accept an \verb+href+ attribute that takes an \uri
specifying a hyperlink same as in the
\hlink{http://dev.w3.org/html5/spec/Overview.html}{\html} element,
namely, \verb+<a>+. So this has an implication in the current scheme of
\texht. If readers provide a few examples of how they want to include
links in a math fragment in \latex, it will help largely to make a
stable and suitable conversion scheme for \texht.

\section{Line breaking}

Previous standards of \mathml had the same limitations as \tex in the
matter of breaking long equations. In \tex, it is easy for an author to
break equations while authoring content taking into consideration of the
margin constrains and paper sizes. However, for web content, margin and
line-length are irrelevant when the window sizes are unknown or
potentially liable to dynamic changes. Therefore, new \mathml 3
specifies a line breaking model and introduces several new attributes to
control properties of line breaks and alignments.

This has deeper implications on the rendering software than on the
engine that creates \mathml.  However, since the specification allows
newer properties for line breaks and alignments, we need to take care of
those part in \texht which at the moments obeys what the author has
input in her \latex document.  Since \tex is incapable of determining
the semantic meaning of equations, we have to provide hooks to make
\texht aware of the real line breaks and obligatory line breaks to
adjust with page sizes.  A better option will be to make use of
\verb+breqn+ package while authoring and \texht to provide necessary
extra functionality to take care of
\hlink{http://ctan.org/pkg/breqn}{\texttt{breqn}}
features in combination with \mathml's newer goodies.

\section{Image inclusion}

The usage of \verb+<glyph>+ element in the previous \mathml
specification has been deprecated now. \verb+<glyph>+ element was used
to specify characters from non-standard fonts that do not correspond to
Unicode code points. Since the usage becomes difficult in a web context,
as one needs to make sure that the font is available, \verb+<mglyph>+
has been extended with \verb+src+ attribute. \verb+<mglyph>+ now acts as
a general image reference element in \mathml much like \html's
\verb+<img>+ element.

\section{Elementary math layout}

Elementary school math layout is difficult to typeset even with \tex.
Usually people adopt a tabular layout with a lot of complex macros to
control space and rules around various number segments. Previous \mathml
standards provide elementary school math layout, much like in \tex, with
a lot of tabular structure which obstructs rendering of math other than
visual, particularly audio, because of the interference of tabular
elements.  \mathml 3 provides better layout elements and scheme which
help to specify even borrows and carries and many different kind of
layout.

\begin{verbatim}
  435.3
   ______
  3)1306
    12
    __
     10
      9
     __
      16
      15
      __
       1.0
         9
         _
         1
\end{verbatim}

\subsection{MathML version of long division}

Given below is the \mathml 3 coding of the above long division. 

\begin{verbatim}
<mlongdiv longdivstyle="lefttop">
  <mn> 3 </mn>
  <mn> 435.3</mn>
  <mn> 1306</mn>
  <msgroup position="2" shift="-1">
    <msgroup>
      <mn> 12</mn>
      <msline length="2"/>
    </msgroup>
    <msgroup>
      <mn> 10</mn>
      <mn> 9</mn>
      <msline length="2"/>
    </msgroup>
    <msgroup>
      <mn> 16</mn>
      <mn> 15</mn>
      <msline length="2"/>
      <mn> 1.0</mn> 
    </msgroup>
    <msgroup position='-1'>
       <mn> 9</mn>
      <msline length="3"/>
      <mn> 1</mn>
    </msgroup>
  </msgroup>
</mlongdiv>
\end{verbatim}

\section{Directionality}

\ifvmode\leavevmode\fi
\mathml 3 provides support for right to left writing/reading direction
whereas previous versions didn't have this functionality. The new
specifications provide the user to control both writing directions of
the identifiers within a formula and layout direction of the formula
itself separately. I am not sure, if this has any influence on the
behavior of current version of \texht which is based on \verb+dvi+ files
generated by \tex/\latex while support for \verb+xdv+, extended dvi
generated by Xe\tex, is still very basic which supports
multi-directional scripts.

\ifx\texhtstandalonedoc1 \end{document}\fi
