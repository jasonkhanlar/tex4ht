% $Id: tex4ht-configure1.tex 131 2014-10-19 18:08:53Z karl $
% Copyright 2009-2013 CV Radhakrishnan.
% Released under LPPLv1.3+.
% 
% TeX4ht's \Configure command, part 1.

\ifx\texhtstandalonedoc1
\documentclass[a4paper]{article}
\usepackage{xspace,graphicx,blog}
\begin{document}
\section{Introduction}
\else
\chapter{\TeX4ht \textbackslash{}Configure}
\fi

The \Verb=\Configure= command is the most powerful user command in
\texht system.  It helps to add various kinds of hooks to insert
target markup code into the desired locations in and around various
types of content.

\section{Introduction to \textbackslash{}Configure}

For instance, take the example of
\Verb=\section{...}= in \latex. A typical example will be:
\begin{verbatim}
  \section{Introduction}
\end{verbatim}
\Verb=\Configure= for \Verb=\section= provides four (4) hooks in the
following style:
\begin{verbatim}
  \Configure{section}
     {<beginning of section>}  {<end of the section>}
     {<before section heading>}{<after section heading}
\end{verbatim}
Suppose our target markup requires the following pattern of markup for
a section heading:
\begin{verbatim}
  <section id="sec1">
   <title>Introduction</title>
    <para>
          ...
    </para>
  </section>
\end{verbatim}
We will accomplish the above target by configuring the section macros
in the following manner:
\begin{verbatim}
  \Configure{section}
   {\EndP\IgnorePar\Tg<section id="sec\thesection">}
   {\EndP\Tg</section}
   {\Tg<title>}
   {\Tg</title>\ShowPar}
\end{verbatim}
The above code will help translate any \latex section heading into the
required markup scheme as provided in the example.  Macros like
\Verb=\EndP=, \Verb=\IgnorePar= and \Verb=\ShowPar= will be discussed
a little later. These are hooks to insert begin and end of paragraph
code into the content appropriately.  Here, in this document, we are
shall list most of the \Verb=\Configure= hooks available in \texht
system and describe briefly how those can be used for configuring most
of the \latex commands popularly used in documents.

\begin{verbatim}
 \Configure{PROLOG}.........1
\end{verbatim}

\begin{tabular}{ll}
\fline\#1 Comma separated list of hooks to appear before \textsm{HTML}.
     Each hook E is declared to be configurable by an
      instruction of the form \Verb+\NewConfigure{E}{1}+. 
  A star \Verb+*+ prefix calls for accumulative configurations.\par

\end{tabular}

\Example
\begin{verbatim}
      \Configure{PROLOG}{VERSION,DOCTYPE,*XML-STYLESHEET}
      \Configure{VERSION}
         {\HCode{<?xml version="1.0"?>}}
\end{verbatim}

\begin{verbatim}
 \Configure{ext}............1
\end{verbatim}
\begin{tabular}{ll}
\fline\#1: default extension name for target files (recorded in
\Verb+\:html+).  Can also be requested through a command line option
\Verb+ext=...+\par

\end{tabular}
\begin{verbatim}
 \Preamble...................0
\end{verbatim}
Records the list of the requested options.  Defined upon entering
   the environment \Verb+\Preamble{...}....\EndPreamble+, to replace the
   earlier version of \Verb+\Preamble+.

\begin{verbatim}
 \ifOption ................. 3
\end{verbatim}
\begin{tabular}{ll}
\fline\#1  Argument to be checked  whether it is a given option.\par

\fline\#2  True part\par

\fline\#3  False part\par

\end{tabular}

\section{Wrapper for the Document}

\begin{verbatim}
 \Configure{DOCTYPE}.........1
 \Configure{HTML}............2
 \Configure{HEAD}............2
 \Configure{@HEAD}...........1
 \Configure{BODY}............2
 \Configure{TITLE+}..........1
 \Configure{TITLE}...........2
 \Configure{@TITLE}..........1
 \Configure{Preamble}........2

    <DOCTYPE>
    <HTML 1>
      <HEAD 1>
         <TITLE 1>
            <@TITLE>
            <TITLE+>
         <TITLE 2>
         <@HEAD>
      <HEAD 2>
      <BODY 1>
      ......
      <BODY 2>
    <HTML 2>
\end{verbatim}

The \Verb+\Configure{@HEAD}{...}+ command is additive, concatenating the
content of all of its appearances.  An empty parameter requests
the cancellation of the earlier contributions.

For instance,
\begin{verbatim}
  \Configure{@HEAD}{A}
  \Configure{@HEAD}{}
  \Configure{@HEAD}{B}
  \Configure{@HEAD}{C}
\end{verbatim}
contributes `BC'.

The \Verb=\Configure{TITLE+}= provides the content for the title,
\Verb+\Configure{TITLE}+ sets the envelop, and
\Verb+\Configure{@TITLE}+ acts as a hook for introducing localized
configurations. As is the case for \Verb+\Configure{@HEAD}+, the
contribution of \Verb+\Configure{@TITLE}+ is also additive.

These configurations should be introduced early enough in the
compilation. For instance, in the case of LaTeX, between \verb+\Preamble+
and \verb+\begin{document}+ of a local configuration file.
\begin{verbatim}
  \Preamble
    %%% here %%%
  \begin{document}
    ...
  \EndPreamble

  \Configure{@BODY}...........1
  \Configure{@/BODY}..........1
\end{verbatim}
  Variants of \Verb=\Configure{@HEAD}= which contribute their content,
   respectively, after \Verb=<body>= and before \Verb=</body>=.
\begin{verbatim}
  \Configure{CutAtTITLE+}.....1
  \Configure{HPageTITLE+}.....1
\end{verbatim}

\begin{tabular}{ll}
\fline   \#1 An insertion just before the content of \Verb+<TITLE>+.
  If \#1 is a one parametric macro, it gets the title content for
   an argument.\par

 \end{tabular}

\section{Support for Sectioning Commands}

\begin{verbatim}
  \Configure{unit-name} ......................4
\end{verbatim}

\begin{tabular}{ll}

\fline   \#1 start\par
\fline    \#2 end\par
\fline    \#3 before title\par
\fline    \#4 after title\par
\end{tabular}
   
\Example

\begin{verbatim}
  \Configure{section}
     {\HCode{<section>}}    {\HCode{</section>}}
     {\HCode{<title>}}      {\HCode{</title>}}
  \ConfigureMark{unit-name}...................1
\end{verbatim}

   Defines a macro \Verb=<unit-name>HMark= to hold the given argument.
   Upon entering the unit, \Verb=\TitleMark= gets the content of this macro.

   Some built-in configurations of \texht require an argument for the
   \Verb=<unit-name>HMark= commands. For safety, these commands should
   always be followed by a (possibly empty) argument.  The argument
   should be a separator between the title mark and its content.

   \Example

\begin{verbatim}
  \Configure{section}
      {}{}
      {\HCode{<h3>}\TitleMark\space}{\HCode{</h3>}}
   \ConfigureMark{section}{\thesection}
\end{verbatim}

\subsection{ToC link}

\begin{verbatim}
  \Configure{toTocLink}.......................2
\end{verbatim}

   Each unit title contains a \Verb=\Link{...}{...}...\EndLink= command.
   The first argument of \Verb=\Link= points to the first table of contents
   referencing the title. The second argument provides an anchor
   for references to the title (mainly from tables of contents).

   The package option \Verb=section+= requests the inclusion of the
   title within the anchor.  Without this option, the link command
   resides between the title mark and its content.

   The \Verb=\Configure{toTocLink}= command is provided for configuring
   the \Verb=\Link= and \Verb=\EndLink=  instructions.  In the default setting,
   when the \Verb=sections+= option is not activated, the \Verb=\Link=
   command is altered to replace its first argument with an empty
   argument.

     \Example

\begin{verbatim}
  \Configure{toTocLink}
     {\Link}
     {\ifx \TitleMark\sectionHMark
        \Picture[\up]{haut.jpg align="right"}%
        \EndLink
        \TitleMark\space
      \else \EndLink \fi
     }
  \def\up{[up]}

  \Configure{toToc}...........................2
\end{verbatim}
\begin{tabular}{ll}

    \fline\#1  unit type

    \fline\#2  desired contents type  (if empty, `unit type' is assumed)

  \end{tabular}
  
\Example 

\begin{verbatim}
  \Configure{toToc}{chapter}{likechapter}
\end{verbatim}
Introduces chapter as likechapter into toc
\begin{tabular}{ll}

\fline \#1  empty: stop adding entries of `unit type' to toc

\fline @: add entries of `unit type' to toc

\fline ?: resume mode in effect before the last stop

\fline \#2  unit type

\end{tabular}
  
\Example
\begin{verbatim}
  \Configure{toToc}{}{chapter}
  \chapter{...}
  \Configure{toToc}{@}{chapter}

  \Configure{writetoc}.........................1
\end{verbatim}
\begin{tabular}{ll}

\fline \#1  Configuration material for the insertion instruction.
        New configurations are added to those request earlier
        by the command.  An empty argument cancels the earlier
        contributions.

\end{tabular}

\begin{verbatim}
  \NoLink.......................1
\end{verbatim}
   Ignore option \Verb=section+= for sections of type \#1.

%
\begin{verbatim}
  \TitleCount
\end{verbatim}
   Count of entries submitted to the \Verb=toc= file
\begin{verbatim}
  \Configure{NoSection}.........2
\end{verbatim}
   Insertions around the parameters of sectioning commands, applied when
   the parameters are not used to create titles for the divisions.

\begin{verbatim}
  \CutAt{#1,#2,#3,...}
\end{verbatim}

\begin{tabular}{ll}

\fspace=20mm

\fline\#1           section type to be placed in a separate web page.

\fline\#2,\#3,...    end delimiting section types, other than \#1, for
          the web pages A \Verb=+= before \#1 requests hypertext
          buttons for the web pages.

\end{tabular}

\Example

\begin{verbatim}
  \CutAt{mychapter,myappendix,mypart}
  \CutAt{+myappendix,mychapter,mypart}
\end{verbatim}
    Cut points at arbitrary points can be introduced by introducing section-like
    commands in a manner similar to
\begin{verbatim}
  \NewSection\mysection{}
  \CutAt{mysection}
  \Configure{+CutAt}.................................3
\end{verbatim}
\fspace=5mm
\begin{tabular}{ll}

\fline     \#1 sectioning type

\fline     \#2 before

\fline     \#3 after \par

\end{tabular}

\smallskip

\noindent     Requests delimiters for the \Verb=\CutAt= buttons of the
specified sectioning type.

%
\Example 

\begin{verbatim}
  \Configure{+CutAt}{mysection}{[}{]}

  \PauseCutAt{#1}
  \ContCutAt{#1}
\end{verbatim}
\begin{tabular}{ll}

\fline    \#1  section type

\end{tabular}

\begin{verbatim}
  \Configure{CutAt-filename} ........................1
\end{verbatim}

\noindent   A 2-parameter hook for tailoring section-based filenames.
   The section type is available through \#1. The section title
   is accessible through \#2.

\Example 

\begin{verbatim}
  \Configure{CutAt-filename}{\NextFile{#1-#2.html}}
\end{verbatim}


\section{Tables of Contents}

Created from the entries collected in the previous compilation within
a \Verb=jobname.4tc= file.
\begin{verbatim}
  \ConfigureToc{unit-name} ......................4
\end{verbatim}
\begin{tabular}{ll}

\fline   \#1 before unit number

\fline   \#2 before content

\fline   \#3 before page number

\fline   \#4 at end

\end{tabular}
\begin{itemize}
\item Empty arguments request the omission of the corresponding field.
\item \Verb=\TocCount=  Specifies the entry count within the
  \Verb= jobname.4tc= file.
\item  \Verb=\TitleCount= Count of entries submitted to the toc file.
\item An alternative to \Verb=\ConfigureToc{unit-name}=:
\end{itemize}
\ifhtml\Tg</ul>\fi

\begin{verbatim}
  \def\toc<unit-name>#1#2#3{<before unit number>#1<before content>#2%
                             <before page number>#3<at end>}
\end{verbatim}

   \Example

\begin{verbatim}
  \ConfigureToc{section}
     {}
     {\Picture[*]{pic.jpg width="13"  height="13"}~}
     {}
     {\HCode{<br />}}
\end{verbatim}


\begin{verbatim}
  \Configure{TocLink}..................4
\end{verbatim}
   Configures the link offered in the third arguments of \Verb=\ConfigureToc=.

   \Example   
\begin{verbatim}
  \Configure{TocLink}{\Link{#2}{#3}#4\EndLink}

  \TocAt{#1,#2,#3,...}
\end{verbatim}
\begin{tabular}{ll}
\fspace=20mm
\fline    \#1           section type for which local tables of contents
                 \Verb=\Toc=\#1 are requested.

\fline    \#2,\#3,...    sectioning types to be included in the tables
of contents.\par

\end{tabular}
\medskip

    The non-leading arguments may be preceded by slashes \Verb=/=, in
    which cases the arguments specify end points for the tables.

    The default setting requests automatic insertion of the local
    tables immediately after the sectioning heads.

    A star \Verb=*= character may be introduced, between the  \Verb=\TocAt= and
    the left brace, to request the appearances of the tables of
    contents at the end of the units' prefaces.

    A hyphen \Verb=-= character, on the other hand, disables the automatic
    insertions of the local tables.

    In case of a single argument, the command removes the
    existing definition of \Verb=\Toc#1=.

\Example

\begin{verbatim}
  \TocAt{mychapter,mysection,mysubsection,/myappendix,/mypart}
  \TocAt-{mysection,mysubsection,/mylikesection}
  \section{...}...\Tocmysection
\end{verbatim}

    The definition  of the local table of contents can be redefined
    within \Verb=\csname Toc#1\endcsname=.

\Example

\begin{verbatim}
  \TocAt{section}
  \def\Tocsection{\TableOfContents[section]}

  \Css{div.sectionTOCS {
                      width : 30%;
                      float : right;
                 text-align : left;
             vertical-align : top;
                margin-left : 1em;
                  font-size : 85%;
           background-color : #DDDDDD;
      }}
\end{verbatim}

\Example 

Table of content before the section title.

\begin{verbatim}
  \Configure{section}{}{}
     {\Tocsection \let\saveTocsection=\Tocsection
      \def\Tocsection{\let\Tocsection=\saveTocsection}%
      \ifvmode \IgnorePar\fi \EndP\IgnorePar
      \HCode{<h3 class="sectionHead">}\TitleMark\space\HtmlParOff}
     {\HCode{</h3>}\HtmlParOn\ShowPar \IgnoreIndent \par}
\end{verbatim}

\begin{verbatim}
  \Configure{TocAt}......................2
  \Configure{TocAt*}.....................2
\end{verbatim}
\fspace=5mm
\begin{tabular}{ll}

\fline   \#1 before the tables of contents

\fline   \#2 after the tables of contents

\end{tabular}

\section{Navigation Links for Sectioning Divisions}

\begin{verbatim}
  \Configure{crosslinks}.....................8
\end{verbatim}
\begin{tabular}{ll}

\fline   \#1  left delimiter

\fline    \#2  right delimiter

\fline    \#3  next

\fline    \#4  previous

\fline    \#5  previous-tail

\fline    \#6  front

\fline    \#7  tail

\fline    \#8  up\par

\end{tabular}
\medskip

The content to be displayed in the pointers.

\begin{verbatim}
  \Configure{crosslinks*}.................1--7
\end{verbatim}

  Links to be included and their order. Available
  options: \Verb=next=, \Verb=prev=, \Verb=prevtail=, \Verb=tail=,
\Verb=front=, \Verb=up=. The last argument must be empty.

\subsubsection*{Default:}

\begin{verbatim}
  \Configure{crosslinks*}
     {next}
     {prev}
     {prevtail}
     {tail} {front}
     {up}
     {}
\end{verbatim}

\begin{verbatim}
  \Configure{crosslinks+}.....................4
\end{verbatim}
\begin{tabular}{ll}

\fline   \#1  before top menu

\fline  \#2  after top menu

\fline   \#3  before bottom menu

\fline   \#4  after bottom menu
\par

\end{tabular}
\medskip

  The top cross links are omitted, if both \Verb=#1= and \Verb=#2= are empty.
  The bottom cross links are omitted, if both \Verb=#3= and \Verb=#4= are empty.

\begin{verbatim}
  \Configure{next}.....................1
\end{verbatim}

\begin{tabular}{ll}

\fline    \#1  the anchor of the next button of the front page.
\par

\end{tabular}
\medskip

    \textbf{Default:} The value provided in \Verb=\Configure{crosslinks}=.

\begin{verbatim}
  \Configure{next+}.............................2
\end{verbatim}

\begin{tabular}{ll}

\fline   \#1  before the next button of the front page, when the \Verb=next=
       option is active.

\fline   \#2  after the button
\par

\end{tabular}

\medskip

\textbf{Default:} The values provided in \Verb=\Configure{crosslinks}=.

\begin{verbatim}
  \Configure{crosslinks:next}..................1
  \Configure{crosslinks:prev}..................1
  \Configure{crosslinks:prevtail}..............1
  \Configure{crosslinks:tail}..................1
  \Configure{crosslinks:front}.................1
  \Configure{crosslinks:up}....................1
\end{verbatim}
\begin{tabular}{ll}

\fline   \#1 local configurations for the delimiters and hooks

\end{tabular}

\begin{verbatim}
  \Configure{crosslinks-}.....................2
\end{verbatim}

   Asks to show linkless buttons with the following insertions.
\begin{tabular}{ll}

\fline      \#1  before

\fline      \#2  after
\par

\end{tabular}
\medskip

   The default values are used, if both \Verb=#1= and \Verb=#2= are empty.

\Example

\begin{verbatim}
  \Configure{crosslinks-}{}{}

  \Configure{crosslinks-}
      {\HCode{<span class="hidden">}[}
      {]\HCode{</span>} }
  \Css{span.hidden {visibility:hidden;}}
\end{verbatim}

\section{Paragraphs}

\begin{verbatim}
  \Configure{HtmlPar}..........4
\end{verbatim}
\begin{tabular}{ll}

\fline   \#1 content at the start non-indented paragraphs

\fline   \#2 content at the start indented paragraphs

\fline   \#3 insertion into \Verb=\EndP=, at the start of non-indented
paragraphs

\fline   \#4 insertion into \Verb=\EndP=, at the start of indented paragraphs

\end{tabular}
\begin{verbatim}
  \HtmlParOff
  \HtmlParOn

  \IgnorePar     Asks to ignore the next paragraph
  \ShowPar       Asks to take into account the following paragraphs

  \IgnoreIndent  asks to ignore indentation in the next paragraph
  \ShowIndent    asks to check indentation in the following paragraphs

  \SaveEndP      Saves the content of \EndP, and sets it to empty content
  \RecallEndP

  \SaveHtmlPar
  \RecallHtmlPar
\end{verbatim}

  \Example
\begin{verbatim}
  \Configure{@BODY}
     {\ifvmode \IgnorePar\fi \EndP
      \HCode{<div>}\par\ShowPar}
  \Configure{@/BODY}
     {\ifvmode \IgnorePar\fi \EndP
      \HCode{</div>}}
\end{verbatim}

\ifx\texhtstandalonedoc1 \end{document}\fi
